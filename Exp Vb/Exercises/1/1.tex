\documentclass[exb]{exercise_5.0}

\deadline{17.10.2024}
\pagenumbering{arabic}

\begin{document}

\section{LEP und LHC}
{\it Am CERN in Genf wurde von 1989 bis 2000 der Elektron-Positron Speicherring LEP
betrieben. Dabei wurden Elektronen und Positronen bis auf eine Energie von 100 GeV
beschleunigt und zur Kollision gebracht. In demselben Tunnel, mit 26.695 km Umfang,
wurden ab 2008 an dem LHC Proton-Proton Kollisionen untersucht. Die Protonen haben
dabei bis heute eine Maximalenergie von 6.8 TeV erreicht.}

\subsection
{Welche Strukturen können mit der Schwerpunktsenergie von LEP bzw. LHC auf-
gelöst werden? Nehmen Sie für LHC an, dass sich die Strahlenergie gleichmäßig auf die drei Quarks im Proton aufteilt.}

\dottedlinett

Elektron und Positron haben die gleiche Masse, daher entspricht das Schwerpunktsystem dem Laborsystem und die Schwerpunktsenergie ist einfach
\begin{align*}
    E_S &= 2 E\sub{lab} = \SI{200}{GeV}.
\end{align*}

Die kleinste aufnehmbare Struktur von LEP ist in der Größenordung von 
\begin{align*}
    \lambda\sub{LEP} &= \frac{hc}{E_S}\\
    &= \SI{6.20e-18}{m}
\end{align*}

Für die Proton-Proton Kollision am LHC sind noch kleinere Strukturen möglich:
\begin{align*}
    \lambda\sub{LHC} &= \frac{hc}{E_S} = \frac{hc}{2\cdot 6.8\u{TeV}}\\
    &\approx \SI{1.82e-19}{m}
\end{align*}

\subsection
{Ein Protonstrahl im LHC besteht aus 2808 einzelnen Paketen, von denen jedes
$115\E9$ Protonen enthält. Wie groß ist die gespeicherte Energie in Joule in jedem Protonstrahl? Welche Masse hätte ein ICE bei $v = 350$ km/   h bei gleicher kinetischer Energie?}

\dottedlinett

Die Energie eines Strahls ist: 
\begin{align*}
    E\sub{Strahl} &= 2808\cdot 115\E9\cdot 6.8 \u{TeV}\\
    &\approx \SI{351}{MJ}
\end{align*}

Damit gilt für die Masse des ICE:
\begin{align*}
    E\sub{Strahl} &= E\sub{kin} 
    = \frac12 m v^2 \\ 
    m &= \frac{2E\sub{Strahl}}{v^2}\\
    &\approx 74.4\u{t
    }
\end{align*}

\subsection{Wieviel Zeit vergeht zwischen der Kollision von zwei Proton-Paketen in einem der LHC-Detektoren? Wie weit können Teilchen in dieser Zeit durch die Teilchendetektoren fliegen? Vergleichen Sie dies mit den Abmessungen von ATLAS und CMS.}

\dottedlinett

Die Zeit zwischen Kollisionen ist
\begin{align*}
    \Delta t &= \frac{L}{Nv} \approx \frac{L}{Nc} \\
    &\approx 31.7\u{ns}
\end{align*}
Laut Wikipedia sollten es \(\SI{25}{ns}\) sein. Die Teilchen fliegen in dieser Zeit:
\begin{align*}
    \Delta s &\approx c\Delta t\\ 
    &\approx 9.5 \u m 
\end{align*}

Sowohl ATLAS als auch CMS sind mit Längen von jeweils 46m und 21m deutlich länger als die Strecke, die sich zwischen zwei Teilchenpaketen befindet. 

\section{Natürliche Einheiten}
\subsection{Drücken Sie die Gravitationskonstante in natürlichen Einheiten (1/GeV$^2$) aus. Geben Sie die zugehörige Massen-, Längen- und Zeitskala in SI-Einheiten an. Diese Skala bezeichnet man als Planck-Skala und sie legt die Grenzen der bisher bekannten physikalischen Gesetze fest.}

\dottedlinett

Ziel ist es die nur Gravitationskonstante durch Elektronenvolt und Konstanten auszudrücken, welche in den natürlichen Einheiten gleich eins gesetzt sind. Die Betrachtung der Einheiten liefert: 
\begin{align*}
    G &= \cG\\
    [G] &= \frac{L^3}{M T^2} \peq[\u{eV}^\alpha\hbar^\beta c^\gamma]\\
    \implies &\begin{cases}
        \u{kg}:\quad  -1 = \alpha+\beta\\
        \u{m}:\quad \, 3 = 2\alpha+2\beta+\gamma\\
        \u{s}:\quad \ \ -2 = -2\alpha - \beta - \gamma
    \end{cases}
    \implies \begin{cases}
        \alpha = -2\\
        \beta = 1\\
        \gamma = 5\\
    \end{cases}\\
    \implies G &= \hbar c^5 \cdot 6.71\E{-57} \ufrac{1}{eV^2}\\
    G\sub{nat} &= 6.71\E{-57} \ufrac{1}{eV^2}
\end{align*}
Die dazugehörigen Massen-, Längen- und Zeitskalen in SI-Einheiten sind:
\begin{table}[H]
\centering
\begin{tabular}{@{}lll@{}}
    \toprule
    {\bf Größe} & {\bf Formel} & {\bf SI-Wert} \\
    \midrule
    Masse & \(\sqrt{\frac{\hbar c}{G}}\) & \(2.18\E{-8}\u{kg}\)\\
    Länge & \(\sqrt{\frac{\hbar G}{c^3}}\) & \(1.62\E{-35}\u m\)\\
    Zeit & \(\sqrt{\frac{\hbar G}{c^5}}\) & \(5.40\E{-44}\u s\)\\
    \bottomrule
\end{tabular}
\end{table}

\subsection{Wie groß ist das Verhältnis von Gravitationskraft zu Coulomb-Kraft von zwei Protonen im Abstand $r$?}

\dottedlinete

\begin{align*}
    \frac{F_G}{F_C} 
    &= \frac{G \frac{m_p^2}{r^2}}{\frac{1}{4\pi\varepsilon_0}\frac{e^2}{r^2}}
    = 4\pi G m_p^2 \varepsilon_0 / e^2\\
    &\approx 8.094\E{-37}
\end{align*}

\section{HERA}
{\it An dem HERA Beschleuniger am DESY in Hamburg wurden von 1992 bis 2007 Elektron-Proton Kollisionen untersucht. Dabei wurden Elektronen auf eine Energie von 27.5 GeV und Protonen auf eine Energie von 920 GeV beschleunigt. Berechnen Sie die Schwerpunktsenergie $E_{CM}$.}

\dottedlinett

Für beide Teilchen ist die kinetische Energie mindestens um drei Größenordnungen größer als die Ruheenergie, womit ultra hochrelativistische Näherungen wie $p\approx \pm E$ gerechtfertigt sind.

\begin{align*}
    s &= (E_e + E_p)^2 - (p_e + p_p)^2\\
    &\approx (E_e + E_p)^2 - (E_e - E_p)^2\\
    \sqrt s&= 2 \sqrt{E_e E_p} \\
    &\approx 318 \u{MeV}
\end{align*}

\section{Relativistische Kinematik - Pionzerfall}
{\it Ein Pion zerfällt in Ruhe in ein Myon und ein Myon-Neutrino: $\pi^+ \to \mu^+\nu_\mu$. Berechnen Sie aus Energie- und Impulserhaltung die Geschwindigkeit ($\nu_\mu/c$) des $\mu^+$
unter der Annahme, dass das Neutrino $\nu_\mu$ masselos ist.\\[1ex]
Tipp: Berechnen Sie dazu zunächst $\abs{p_\mu}$ aus dem Ansatz $p_\pi=p_\mu + p_\nu$ der Vierervektoren.}

\dottedlinett

Aus der Viererimpuls-Erhaltung folgt:
\begin{align*}
    (E_\pi,  p_\pi) &= (E_\mu + E_\nu, p_\mu + p_\nu)\\
    (m_\pi,0) &= \hug{\sqrt{m_\mu^2 + p_\mu^2} + p_\nu, p_\mu + p_\nu}\\
    \\
    \implies p_\nu &= - p_\mu\\
    \implies m_\pi &= \sqrt{m_\mu^2 + p_\mu^2} - p_\mu\\
    (m_\pi + p_\mu)^2 &= m_\mu^2 + p_\mu^2\\
    m_\pi^2 + 2m_\pi p_\mu &= m_\mu^2 \\
    \implies p_\mu &= \frac{m_\mu^2 - m_\pi^2}{2m_\pi } \\
    \implies p_\nu &= \frac{m_\pi^2 - m_\mu^2}{2m_\pi }
\end{align*}

Aus dem Impuls lässt sich nun die Geschwindigkeit berechnen:
\begin{align*}
    p_\mu &= \gamma m_\mu v\\
    p_\mu^2 &= \frac{\beta^2}{1-\beta^2} m_\mu^2\\
    0 &= \beta^2( m_\mu^2 + p_\mu^2) - p_\mu^2\\
    \beta &= \frac{p_\mu}{\sqrt{m_\mu^2 + p_\mu^2}}\\
     &= \frac{\frac{m_\mu^2 - m_\pi^2}{2m_\pi }}{\sqrt{m_\mu^2 + \hug{\frac{m_\mu^2 - m_\pi^2}{2m_\pi }}^2}}\\
    &= \frac{m_\mu^2 - m_\pi^2}{\sqrt{4m_\pi^2 m_\mu^2 + \hug{m_\mu^2 - m_\pi^2}^2}}\\
    &= \frac{m_\mu^2 - m_\pi^2}{\sqrt{4m_\pi^2 m_\mu^2 + m_\mu^4 - 2m_\mu^2 m_\pi^2 + m_\pi^4}}\\
    &= \frac{m_\mu^2 - m_\pi^2}{\sqrt{m_\mu^4 + 2 m_\mu^2 m_\pi^2 + m_\pi^4}}\\
    &= \frac{m_\mu^2 - m_\pi^2}{m_\mu^2 + m_\pi^2}\\
    &\approx 0.422\\
    v &= c\beta \approx 1.26\E{8}\ufrac ms 
\end{align*}

\end{document}