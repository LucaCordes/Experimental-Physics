\documentclass[twocolumn]{summery}
\title{Experimentalphysik III - Zusammenfassung}

\begin{document}
\maketitle
\tableofcontents

\section{Geometrische Optik}
\subsection{Fermat's Prinzip}
Die geometrische Optik lässt sich mathematisch elegant beschreiben wenn man den Lichtweg 
\(L = \int \abs{\vec r(t)}\cdot n(\vec r (t)) \dt\) definiert. Er ist der normale Weg, gewichtete 
mit dem lokalen Brechungsindex.
Das Licht nimmt immer den Weg, der den Lichtweg extremal werden lässt.
Zur Erinnerung: Es gilt \(n = \frac{c}{v}\)

Es Weg des Lichts kann daher formal mithilfe der Euler-Lagrange Gleichungen beschrieben werden:
\begin{align*}
    \ddt \partiald{\mathcal L}{\vdot x} = \partiald{\mathcal L}{\vec x} \with \mathcal L = \abs{\vec r(t)}\cdot n(\vec r (t))
\end{align*}

\subsection{Snell's Gesetz}
Reist ein Lichtstrahl von einem Medium mit Brechungsindex \(n_1\) in ein zweites mit 
Brechungindex \(n_2\), wird er gebrochen. Der Winkel kann mithilfe von Snell's Gesetz 
berechnet werden:
\begin{align*}
    \frac{\sin\beta}{\sin\alpha} &= \frac{n_a}{n_b}
\end{align*}

\end{document}