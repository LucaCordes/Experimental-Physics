\documentclass[ex]{exercise}

\deadline{07.11.2023}

\begin{document}

\section{Zweilinsensystem}
Welchen Abstand müssen zwei Sammellinsen von je 10 cm Brennweite haben, damit ihre 
Gesamtbrennweite \(f=8 \u{cm}\) ist?

\dottedline

\begin{align*}
    D' &= D_1 + D_2 - d D_1 D_2\\
    &= 2 D - d D^2\\
    d &=\frac{2D - D'}{D^2}\\
    &\approx \frac{2\cdot \frac 1{10\u{cm}} - \frac1{8\u{cm}}}{\frac1{10^2\u{cm}^2}}\\
    &\approx 7.5\u{cm}
\end{align*}

\section{Vorsatzlinse für Kamera}
Eine Kamera mit Brennweite $f_1 = 5 \u{cm}$ soll mit einer Vorsatzlinse versehen werden, sodass eine 
Briefmarke in natürlicher Größe erscheint, wenn die Kamera auf \(\infty\) eingestellt wird.
Wie groß ist die Brennweite $f_2$ der Vorsatzlinse?

\dottedline\vspace{-0.2cm}

Die Briefmarke erscheint in natürlicher Größe, wenn das gesamt System aus Objektiv und Vorsatzlinse
das Licht ungebrochen durchlässt \(\implies D' = 0\).
Für dünne Linsen kann die Brechkraft einfach auf addiert werden:
\begin{align*}
    D' &= D_V + D_L=0\\
    f_V &= - f_L\\
\end{align*}

\section{Vergrö{\ss}erung am Kepler'schen Fernrohr}
Ein Kepler’sche Fernrohr besteht aus einem Objektiv mit Brennweite $f_{\te{obj}}$ 
und einem Okular mit Brennweite $f_{\te{okl}}$ die sich den selben Brennpunkt teilen.

\subsection{}
Bestimmen Sie für einen Strahl der in der Höhe h und Winkel \(\alpha\) das Objektiv trifft die 
Abbildungsmatrix. Das Objektiv ist eine plankonvexe Linse mit dem Radius von $r_1 = 24 \u{cm}$, 
das Okular ist eine bikonvexe Linse mit Biegeradius $r_2 = 12 \u{cm}$. 
Beide Linsen sind aus Kronglas mit einem Brechnungsindex von $n_L = 1.6$ gefertigt.

\dottedline

\begin{align*}
    D_1 &= (n_L-n_0)\hug{\frac1{r_{11}} + \frac{1}{r_{12}}}
    \approx (1.6-1)\hug{\frac1{24\u{cm}} + \frac{1}{\infty \u{cm}}}
    \approx \frac1{40} \u{dpt}\\
    D_2 &= (n_L-n_0)\hug{\frac1{r_{21}} + \frac{1}{r_{22}}}
    \approx (1.6-1)\hug{\frac1{12\u{cm}} + \frac{1}{12 \u{cm}}}
    \approx \frac1{10} \u{dpt}
\end{align*}
Abbildungsmatrix für das ganze System:
\begin{align*}
    M &= M_{L_2}\cdot M_{T}\cdot M_{L_1}\\
    &=
    \begin{pmatrix}
        1&0\\
        -D_2&1\\
    \end{pmatrix}
    \begin{pmatrix}
        1&\frac{d}{n_0}\\
        0&1\\
    \end{pmatrix}
    \begin{pmatrix}
        1&0\\
        -D_1&1\\
    \end{pmatrix}\\
    &= \begin{pmatrix}
        1 & \frac{d}{n_0}\\\
        -D_2 & 1 - D_2 \frac{d}{n_0}\\
    \end{pmatrix}
    \begin{pmatrix}
        1&0\\
        -D_1&1\\
    \end{pmatrix}\\
    &= \begin{pmatrix}
        1 - D_1 \frac{d}{n_0}& \frac{d}{n_0}\\\
        D_1 D_2 \frac{d}{n_0} - D_2 - D_1& 1 - D_2 \frac{d}{n_0}\\
    \end{pmatrix}\\
    &\approx \begin{pmatrix}
        -\frac14& 50\u{cm}\\
        0 & -4\\
    \end{pmatrix}\with 
    \begin{cases}
        d = \frac1{D_1}+\frac1{D_2}=50\u{cm}\\
        n_0 \approx 1
    \end{cases}
\end{align*}

\subsection{}
Mit der Abbildungsmatrix:
\begin{align*}
    S_2 &= M \cdot S_1 = M \cdot \binom{0}{\alpha}\\
    \implies \beta &= -4\alpha\\
    V &= \frac{\tan\beta}{\tan\alpha}
    = \frac{\tan(-4\alpha)}{\tan\alpha}
    \approx -4 \since \tan\alpha =\alpha + \bigO{\alpha^3}
\end{align*}
Mithilfe die Geometrie:
\begin{align*}
    V &= \frac{\tan\beta}{\tan\alpha}
    = \frac{b_z / F_2}{-b_z / F_1}
    = - \frac{D_2}{D_1} 
    = -4
\end{align*}

\section{Öffnungsfehler plankonvexe Linse}

\end{document}