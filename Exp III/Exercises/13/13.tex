\documentclass[ex,minted]{exercise_3.1}

\deadline{23.01.2024}

\begin{document}

\section{Unschärferelationen}
\begin{align*}
    \Delta A \cdot \Delta B &\ge \frac12 \abs{\left\langle{\psi\abs{\bug[]{\hat A,\, \hat B}}\psi}\right\rangle}
\end{align*}
\subsection{}
\begin{align*}
    \Delta p \cdot \Delta x &\ge \frac\hbar2\\
    \Delta x &\ge \frac{\hbar}{2m_e\Delta v } \\
    &\approx \frac{\chbar}{2\cdot \cme \cdot 14.5 \ufrac ms}\\
    &\approx 4.01 \E{-6} \,\mathrm  m
\end{align*}

\subsection{}
\begin{align*}
    \Delta E \cdot \Delta t &\ge \frac{\hbar}{2}\\
    \Delta t &\ge \frac{\hbar}{2 \Delta E}\\
    &\approx \frac{\chbar}{2 \cdot 0.2 \,\u{eV}}\\
    &\approx 1.65\E{-15} \u s
\end{align*}

\subsection{}
\begin{align*}
    L &= \snorm L \\
    &= \sqrt{L_x^2 + L_y^2 + L_z ^2}\\
    \Delta L &=\frac{\sum_iL_i \Delta L_i}{\sqrt{L_x^2 + L_y^2 + L_z ^2}}\\
    &=\frac{\v L \cdot \Delta \v L}{\snorm {\v L}}\\
    &= \v e_L\cdot \Delta \v L\\
    &\ge \Delta L_z
\end{align*}
Die \(z\)-Komponente des Drehimpulses kann im Prinzip beliebig genau gemessen werden.

\section{Laser Energieniveau}
\subsection{}
\begin{align*}
    E_{\te{Ph}} &= \frac{hc}{\lambda}\\
    \lambda &= \frac{hc}{ E_{\te{Ph}}}\\
    &\approx \frac{\ch \cdot \cc}{1.79 \u{eV}}\\
    &\approx 693\u{nm} \note[rotes Licht ]
\end{align*}

\subsection{}
\begin{align*}
    E &= \frac{h c}{\lambda}\\
    \overset{\substack{\te{Gauß'sche}\\\te{Fehlerfortpflanzung}}}\implies\Delta E &= \frac{h c}{\lambda^2}\Delta \lambda
    \note\tt{kein Minus, da Unsicher-}{heiten per Def. positiv}\\
    &= \frac{h c E^2}{(hc)^2}\Delta \lambda\\
    &= \frac{E^2}{hc}\Delta \lambda\\
    &\approx \frac{1.79^2\u{eV^2}}{\ch\cdot \cc} \cdot 0.53\u{nm}\ce\\
    &\approx 2.19 \E{-22} \u J
\end{align*}

\section{Korrelation von Spins}
\subsection{}
\begin{align*}
    \tug{\hat O_1} &= \tug{\uparrow_1 \mid \hat O_1 \mid \uparrow_1}\\
    &= \int\d\alpha \binom10 \hug{\sigma_x\cos\alpha + \sigma_y\sin\alpha}\binom 10\\
    &= \int\d\alpha \binom10 \begin{pmatrix}
        0 & e^{-i\alpha}\\
        e^{i\alpha} &0 \\
    \end{pmatrix}\binom 10\\
    &= \int \d\alpha\binom10 \binom0{e^{i\alpha}}\\
    &= 0\\
    \\
    \tug{\hat O_2} &= \dots =0 \note[analog zu oben] 
\end{align*}

\subsection{}
\begin{align*}
\end{align*}

\subsection{}
Das Minuszeichen entspricht der Tatsache, dass es bekannt ist, dass die Spins beiden Elektronen
antiparallel sind. Obwohl die gemessenen Drehimpulse zufällig verteilt sind, gibt einen 
die Messung des einen Elektron somit durch die Korrelation 
Informationen über die noch ausstehende Messung des anderen Elektron.
Wenn der eine Spin immer entgegengesetzt des anderen ist, dann ergibt positiv mal negativ immer 
ein negatives Produkt.
 
\section{Zustandsenergien endlicher Potenzialtopf}
\inputpy{Python Code:}{13.py}

\end{document}