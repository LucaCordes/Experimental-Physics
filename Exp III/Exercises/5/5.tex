\documentclass[ex]{exercise}

\deadline{14.11.2023}

\begin{document}

\section{Frauenhofer-Achromat}
\begin{align*}
    &\begin{cases}
        0=f_1\nu_1 + f_2\nu_2\\
        D=D_1+D_2\\
    \end{cases}\\
    0&= \frac{\nu_1}{D-D_2} + f_2\nu_2\\
    0&= \nu_1 + \hug{\frac{D}{D_2}-1}\nu_2\\
    \frac{1}{D_2} &= \frac{1}{D}\hug{1-\frac{\nu_1}{\nu_2}}\\
    \\
    f_2 &= f\hug{1-\frac{\nu_1}{\nu_2}}\\
    &\approx 50\u{mm}\cdot  \hug{1-\frac{63.4}{27.5}}\\
    &\approx -65.3\u{mm}\\
    \\
    f_1 &= f_2\frac{\nu_2}{\nu_1}\\
    &\approx 65.3\u{mm}\cdot \frac{27.5}{63.4}\\
    &\approx 28.3\u{mm}
\end{align*}

Bei den Linsen mit Brennweiten \(f_1=28.3\u{mm}\) und \(f_2=-65.3\u{mm}\)
handelt es sich jeweils um eine konkave Linse/Sammellinse und 
eine konvexe Linse, also eine Zerstreuungslinse.

\section{Thermische Strahlung}
Wie gro{\ss} sind die Wellenlängen \(\lambda_{\max}\), bei 
der die folgenden Quellen genähert als Schwarzkörper,
ihre maximale wellenlängenabhängige Leistung abstrahlen?

\dottedlinett

Nach dem Wien'schen Strahlungsgesetz gilt für die Energiedichte, abhängig von der Wellenlänge:
\begin{align*}
    M_E(\lambda) &= \frac{c_1}{\lambda^5} \frac{1}{e^{\frac{c_2}{\lambda T}}}
        = \frac{2\pi hr c^2}{\lambda^5} \frac{1}{e^{\frac{h c}{k_B}\frac{1}{\lambda T}}}
\end{align*}
Für ein Maximum muss gelten:
\begin{align*}
    0 &= \deriv{M_E}{\lambda} \\
    &= \frac{2\pi h c^2}{\lambda^5} \cdot \frac{h c}{k_B}\frac{1}{\lambda^2 T}  \frac{1}{e^{\frac{h c}{k_B}\frac{1}{\lambda T}}} 
    -\frac{10\pi h c^2}{\lambda^6}\cdot \frac{1}{e^{\frac{h c}{k_B}\frac{1}{\lambda T}}}\\
    0 &= \frac{h c}{k_B}\frac{1}{T} - 5 \lambda\\
    \lambda_{\max} &= \frac{h c}{5 k_B T}                  
\end{align*}

\subsection{Eine Glühlampe mit Temperatur \(T=3000\u K\)}
\begin{align*}
    \lambda_{\max} &= \frac{h c}{5 k_B T} \approx \frac{\ch \cdot \cc}{5\cdot \ckb \cdot 3000\mathrm K}
    \approx 961 \mathrm{nm}
\end{align*}

\subsection{Ein Mensch mit Körpertemperatur \(T=37^\circ \mathrm C\)}
\begin{align*}
    \lambda_{\max} &= \frac{h c}{5 k_B T} \approx \frac{\ch \cdot \cc}{5\cdot \ckb \cdot \hug{273+37}\mathrm K}
    \approx 9299 \mathrm{nm}
\end{align*}

\subsection{Ein Lagerfeuer mit Temperatur \(T=800^\circ \u C\)}
\begin{align*} 
    \lambda_{\max} &= \frac{h c}{5 k_B T} \approx \frac{\ch \cdot \cc}{5\cdot \ckb \cdot \hug{273+800}\mathrm K}
    \approx 2690 \mathrm{nm}
\end{align*}

\subsection{Eine Atombombenexplosion mit Temperatur \(T=10^7\u K\)}
\begin{align*}
    \lambda_{\max} &= \frac{h c}{5 k_B T} \approx \frac{\ch \cdot \cc}{5\cdot \ckb \E7\mathrm K}
    \approx 0.288 \mathrm{nm}
\end{align*}

\subsection{Die kosmische Hintergrundstrahlung mit Temperatur \(T=1.7\u K\)}
\begin{align*}
    \lambda_{\max} &= \frac{h c}{5 k_B T} \approx \frac{\ch \cdot \cc}{5\cdot \ckb \cdot 2.7\mathrm K}
    \approx 1.07 \mathrm{mm}
\end{align*}

\section{Temperatur der Erde}
\subsection{Zeigen Sie, dass sich die Oberflächentemperatur eines Planeten umgekehrt 
propertional zur Wurzel aus seinem Abstand zur Sonne verhält. Nehmen Sie dafür an, dass sowohl 
die Sonne als auch der Planet als Schwarzerkörper beschrieben werden können,
wobei die Temperatur auf Sonne und Planet überall auf der Oberfläche gleich sein soll.}

\dottedlinett

Geht man von einem statischen Zustand aus, so ist die von der Sonne 
auf die Erde einstrahlende Leistung im Äquilibrium mit der Leistung der
Schwarzerkörperstrahlung der Erde. 
\begin{align*}
    0 &= \eta \cdot \Phi_E(\te{Sonne}) + \Phi_E(\te{Erde})\note \eta :=\tt{Anteil der Sonnenstrahlen}{die auf die Erde treffen}
\end{align*}
Die Schwarzerkörperstrahlung der beiden Himmelskörper lässt sich 
mit dem Stefan-Boltzmann-Gesetz bestimmen:
\begin{align*}
    &= \eta \cdot \sigma A_S  T_S^4 - \sigma A_E T_E^4\\
    &= \frac{2\pi r_E^2}{4\pi r_{SE}^2}\cdot A_S  T_S^4 - A_E T_E^4\\
    &= \frac{r_E^2}{2 r_{SE}^2}\cdot  4\pi r_S^2  T_S^4 - 4\pi r_E^2 T_E^4\\
    T_E^4 &= {\frac12\frac{r_S^2}{r_{SE}^2} T_S^4 }\\
    T_E &= \sqrt{\frac{1}{\sqrt 2}\frac{r_S}{r_{SE}}} T_S
\end{align*} 

\subsection{Welche Temperatur ergibt sich nach diesem Modell für die Erde? Vergleichen Sie mit der tatsächlichen 
Temperatur auf der Erde.}

\dottedlinete

\begin{align*}
    T_E &= \sqrt{\frac{1}{\sqrt 2}\frac{r_S}{r_{SE}}} T_S\\
    &\approx \sqrt{\frac{1}{\sqrt 2}\frac{6.96\E8\u m}{1.50\E{11}\u m}} \cdot 5778\u K\\
    &\approx 331\u K \approx 58.0\c
\end{align*} 
Laut Wikipedia ist die echte durchschnittliche Temperatur der Erde \(15\c\); damit ist 
die Temperatur des Modells recht nah an der Realität. 


\section{Fotometriegrö\ss en}

\begin{enumerate}
    \item Strahlungsphysikalische Grö\ss en:
    \begin{enumerate}
        \item Strahlfluss:\\
        - Die Strahlungsleistung, welche von einem Objekt ausgeht.\\
        - $\Phi_E$, Einheit: \(1\u W\)
        
        \item Strahlstärke:\\
        - Die Strahlungsleistung, die in einen bestimmten Raumwinkel ausgestrahlt wird\\
        - Mathematische Definition: \(I_E = \frac{\d \Phi_E}{\d\Omega}\)\\
        - Einheit: \(1 \ufrac{W}{sr}\)
        
        \item Strahldichte:\\
        - Die Strahldichte, beschreibt Strahlungsleistung in Richtung eines bestimmten Raumwinkels, bezogen 
        auf die projezierte abstrahlende Fläche \\
        - Mathematische Definition: \(I_E = \frac{1}{\cos\phi}\frac{\d \Phi_E}{\d A\d\Omega}\)\\
        - Einheit: \(1 \ufrac{W}{m^2\,sr}\)
    \end{enumerate}
    
    \item Lichttechnische Grö\ss en:
    \begin{enumerate}
        \item Lichtstrom:\\
        - Lichttechnisches Equivalent zum Strahlungsfluss\\
        - Der Lichtstrom gibt an
        wie viel sichtbares Licht von einem Objekt abgestrahlt wird, wobei auch die unterschiedliche 
        Empfindlichkeit des Auges gegenüber verschiedenen Wellenlängen berücksichtigt wird.\\
        - \(\Phi_V\), Einheit: \(1\u{lm}\), gesprochen Lumen
    
        \item Lichtstärke:\\
        - Lichttechnisches Equivalent zur Strahlstärke\\
        - Die Lichtstärke beschreibt die Menge des wahrnehmbaren Lichts in eine bestimmte Raumrichtung\\
        - Mathematische Definition: \(I_V = \frac{\d\Phi_V}{\d\Omega}\)\\
        - Einheit: \(1 \u{cd}\), gesprochen Candela

        \item Leuchtdichte: \\
        - Lichttechnisches Equivalent zur Strahldichte\\
        - Die Lichtdichte, beschreibt den Lichtstrom in Richtung eines bestimmten Raumwinkels, bezogen 
        auf die projezierte abstrahlende Fläche \\
        - Mathematische Definition: \(L_V = \frac{\d\Phi_V}{\d\Omega}\)\\
        - Einheit: \(1 \ufrac{cd}{m^2}\)
    \end{enumerate}
\end{enumerate}
\end{document}