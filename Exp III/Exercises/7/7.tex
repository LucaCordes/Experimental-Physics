\documentclass[ex]{exercise}

\deadline{28.11.2023}

\begin{document}

\section{Beugungsscheiben eines Sterns}
Mit einer Kamera vom Öffungsverhältnis $\frac Df = \frac{1}{2.8}$, mit 
dem Öffnungsdurchmesser $D$ und der Brennweite $f$ , wird
ein Stern fotografiert. Welchen Radius hat das auf dem Fotochip entstehende zentrale Beugungsscheibchen 
$(λ =600 \u{nm})$?

\dottedlinett

Nach dem Rayleigh-Kriterium gilt für die Winkeldistanz bei der 
sich das Beugungsminimum erster Ordnung, des einen Objektes, mit dem
Beugungsmaximum erster Ordnung, des anderen Objektes, überlappt:
\begin{align*}
    \sin\theta_{min} &=  1.22 \frac{\lambda}{D}\\
    r_{min} &=  1.22 \frac{f \lambda}{D}\\
    &\approx 1.22 \cdot 2.8 \cdot 600\u{nm}\\
    &\approx 2050 \u{nm}
\end{align*}

\section{Spektrumüberlappung}
Weisen Sie nach, dass das rote Ende $(\lambda_1 = 700 \u{nm})$ des Spektrums 2. Ordnung eines Beugungsgitters vom violetten
Ende des Spektrums 3. Ordnung $(\lambda_2 = 400 \u{nm})$ überlappt wird

\dottedlinett

\begin{align*}
    \Delta \phi &= 2\pi \frac d\lambda \sin\theta\\
    n &= \frac d\lambda \sin\theta \for[konstr. Interferenz]\\
    \theta_n &= \arcsin\hug{\frac{n \lambda}d}\\
    \\
    \theta_2(\lambda_1) &= \arcsin\hug{\frac{2\cdot 700\u{nm}}{d}}\\
    &= \arcsin\hug{\frac{1400\u{nm}}{d}}\\
    \theta_3(\lambda_2) &= \arcsin\hug{\frac{3\cdot 400\u{nm}}{d}}\\
    &= \arcsin\hug{\frac{1200\u{nm}}{d}}\\
    \\
    \implies \theta_2(\lambda_1) &> \theta_3(\lambda_2)\since \arcsin x \te{ streng monoton steigend}
\end{align*}
Da das Ende des Spektrums 2. Ordnung mit einem grö{\ss}eren Winkel abgestrahlt wird als der 
Anfang des Spektrums 3. Ordnung, ist klar, dass sich die Spektren 2. und 3. Ordnung überlappen.
Bei höheren Ordungen wird die Überlappung noch signifikanter werden.

\section{Laser auf gerader Oberfläche}
Scheint man mit einem optischen Laser auf eine gerade Oberfläche, sieht man ein Scheibchen reflektierten Lichts
(siehe Bild).
Man erkennt, dass dem Intensitätsprofil des Lasers (circa gaussisch) eine granulare Feinstruktur überlagert ist.
Betrachtet ein(e) Brillenträger(in) das Scheibchen ohne Brille in einer Distanz, in der er/sie nicht mehr scharf
sehen kann, verschwimmt das Scheibchen. Die granulare Feinstruktur verschmiert interessanterweise nicht.
(Jemand der üblicherweise keine Brille trägt, beobachtet den gleichen Effekt, wenn er das Scheibchen durch die
Brille jemand anderes beobachtet).\\
Erklären Sie diese Beobachtung qualitativ!

\dottedlinett

Das Laserlicht ist nahezu perfekt kohärent. Wenn es auf die Oberfläche 
trifft, wird es gestreut; von jedem Punkt der Oberfläche geht nun eine Kugelwelle aus,
dessen Phase von der mikroskopische Beschaffenheit der Oberfläche abhängt. 
Da das Licht zu diesem Zeitpunkt immer noch kohärent ist, interferiert es, und es bilden
sich praktisch zufällig im Raum verteilte Intensitätsminima/maxima. 
Dieser Effekt findet unabhängig davon statt, ob das Licht in einer optischen Vorrichtung, wie 
dem Auge, gebrochen wird; Folglich kann der Effekt auch beobachtet werden, wenn die 
Augen des Beobachters nicht vollständig akkommodiert sind.
Zusammenfassend handelt es sich also um ein Beugungs/Interferenzphänomen.


\section{Kohärenz beim Michelson-Interferometer}
Bei Messungen mit einem Michelson-Interferometer wird bei einem der beiden 
Interferometer-Arme der Spiegel bewegt und dabei das Erscheinen und Verschwinden der 
Maxima im Zentrum des Interferenzbildes auf dem Schirm beobachtet. Im Interferenzmeter
werde zunächst Licht der roten Cadmium-Linie mut der Wellenlänge \(\lambda=646.8\u{nm}\)
und der Linienbreite \(\Delta \lambda = 0.0013\u{nm}\) benutzt. Für die Abstrahlung 
gilt für die Frequenz \(\nu\) und die Abstrahldauer \(t\), \(\nu \cdot t=1\)

\subsection{Wie gro{\ss} ist die gesamte Verstellstrecke des Spiegels, innerhalb derer eine 
Interferenzbild zu beobachten ist?}

\dottedlinett

Die Kohärenzzeit ist die Zeit, in der zwei Wellen verschiedener Wellenlänge ihre 
Phasenrelation um \(180^\circ\) wechselt, daher muss gelten:
\begin{align*}
l_c  &= \begin{cases}
    n \lambda_1\\
    (n+1) \lambda_2\\
\end{cases}\\
\implies l_c &= \hug{\frac{ct_c}{\lambda_1} + 1}\lambda_2\\
 &= \frac{\lambda_2}{1-\frac{\lambda_2}{\lambda_1}}\\
&= \frac{\lambda_1\lambda_2}{\lambda_1-\lambda_2}\\
&= \frac{(\lambda + \Delta\lambda/2)(\lambda - \Delta\lambda/2)}{\Delta\lambda}\\
&= \frac{\lambda^2 - \Delta\lambda^2/4}{\Delta\lambda}\\
l_c &\approx \frac{\lambda^2}{\Delta \lambda}\for \Delta \lambda\ll \lambda\\
 &= \frac{(643.8\E{-9})^2}{0.0013\E{-9}}\\
 &\approx 0.319 \u m
\end{align*}
Im Interferemeter lässt sich auf einer gesamten Verstellstrecke von ca. \(0.319 \u m\)
ein Interferenzbild beobachten.

\subsection{Wie gro{\ss} ist sie, wenn man Licht eines Helium-Neon-Lasers mit 
\(\lambda = 632.8\u{nm}\) und einer Frequenzstabilität von \(2\E{-10}\) benutzt?}

\dottedlinett


\section{Fresnelsche Zonenplatte}
Eine Zonenplatte ist eine ebene Glasplatte mit konzentrischen Kreisringen, die abwechselnd lichtdurchlässig und
lichtundurchlässig sind. Die innerste Kreisfläche ist dabei lichtundurchlässig. 
Bestimmen Sie die Radien der Kreisringe so, dass die Platte als symmetrischen bikonvexen Sammellinse wirkt

\subsection{Wie müssen die Radien gewählt werden, wenn Licht der Wellenlänge 600 nm mit einer Brennweite von 50 cm
fokussiert werden soll?}

\dottedlinett

Es wird die Phasendifferenz im Fokuspunkt zwischen einem Lichtstrahl
der durch die optische Achse verläuft, und einem der im Abstand \(r\) von der 
optischen Achse gebeugt wurde, berechnet. Ziel ist es den Kreis in Zonen einzuteilen, in
denen jeweils konstruktive/destruktive Interferenz stattfindet. Für die 
Radien \(r_n\) an denen konstruktive zu destruktiver Interferenz wechselt, muss gelten, dass 
die überlagerte normierte Amplitude gleich eins ist, also weder konstruktive noch destruktive
Interferenz stattfindet:
\begin{align*}
    \phi(r) &= \frac{2\pi \sqrt{f^2 + r^2}}{\lambda}\\
    \\
    n\pi &= \Delta \phi = \phi(r)-\phi(0)\note n\in\N\\
    &= \frac{2\pi \sqrt{f^2 + r_n^2}}{\lambda} - \frac{2\pi f }{\lambda}\\
    r_n &= \sqrt{\hug{\frac{n\lambda}2 + f}^2 - f^2}\\
    &= \sqrt{n\lambda f + \frac{n^2\lambda^2}4}
\end{align*}

\subsection{Vergleichen Sie die chromatische Aberration \(\dd D\lambda\)
dieser Zonenplatte mit der einer Sammellinse aus Flintglas mit
der gleichen Brennweite}

\dottedlinett

Für die chromatische Abberation der Sammellinse aus Flintglas gilt:
\begin{align*}
    D &= \frac{2(n_L(\lambda) - n_0)}{r}\\
    r &= 2f (n_L(600\u{nm}) - n_0)\\
    &= 2\cdot 50\u{cm} \cdot (1.61 - 1)\\
    &= 61 \u{cm}\\
    \dd D\lambda &= \frac{2}{r} \dd{n_L}\lambda\\
    &= - \frac{2}{61 \u{cm}} \cdot 0.97\E5 \u m\inv\\
    &\approx - 31.8 \u{cm}^{-2}\\
\end{align*}

Für die Fresnelsche Linse gilt hingegen:
\begin{align*}
    r_n &= \sqrt{n \lambda f + \frac{n^2}{4}\lambda^2}\\
    &\approx \sqrt{n \lambda f} \since \lambda\ll f\\
    \\
    D &=\frac1f =  \frac{n\lambda}{r_n^2}\\
    \dd D\lambda &= \frac{n}{r_n^2}\\
    &= \frac1{\lambda_0 f}\\
    &\approx 3.33 \u m^{-2}
\end{align*}
Es fällt auf, dass die Brechkraft der Fresnellinse mit steigender Wellenlänge
zunimmt, bei der Linse aus Flintglas andersrun; Au{\ss}erdem ist die 
chormatische Abberation der  
Fresnellinse um mehrere Grö{\ss}enordnungen stärker.

\end{document}