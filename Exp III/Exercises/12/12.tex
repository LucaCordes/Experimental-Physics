\documentclass[ex]{exercise_3.0}

\deadline{16.01.2024}

\begin{document}

\section{Aufenthaltswahrscheinlichkeit des Elektrons im H-Atoms}
Die Wellenfunktion des Elektrons im Grundzustand des Wasserstoffatoms lautet 
\begin{align*}
    \psi(r) &= \frac1{\sqrt{\pi a^3}} e^{-\frac ra}
\end{align*}
Die Aufenthaltswahrscheinlichkeit des Elektrons ist egegen durch 
\begin{align*}
    P(r) = 4\pi r^2 \abs{\psi(r)}^2
\end{align*}
Wie groß ist der wahrscheinlichste Abstand des Elektrons zum Kern?

\dottedlinete

\begin{align*}
    0&\peq \dd Pr\\
    &= \pp{}r \hug{4\pi r^2 \abs{\psi(r)}^2}\\
    &= \pp{}r \hug{4\pi r^2 \frac1{\pi a^3} e^{-\frac {2r}a}}\\
    &= \pp{}r \hug{\frac{4r^2}{a^3} e^{-\frac {2r}a}}\\
    &= \hug{\frac{8r}{a^3} - \frac{8r^2}{a^4}} e^{-\frac {2r}a}\\
    \implies 0 &= -a r + r^2\\
    \implies r &= a \for a> 0 
\end{align*}

\section{Endlicher Potentialtopf}
Ein endlicher Potentialtopf sei gegeben über 
\begin{align*}
    V(x) &= \begin{cases}
        0 \,\,\,\for x\in[-a,a]\\
        V_0 \telse
    \end{cases} \ \  \ \with V_0>0
\end{align*}
Bestimmen Sie die allgemeine Lösung der stationären Schrödingergleichung für ein Teilchen mit der Energie \(E\)
mit \(0<E<V_0\) im Potential \(V(x)\). Benutzen Sie als Lösungsansatz
\begin{align*}
    \psi_1 (x) &= \alpha_1 e^{\kappa x} + \alpha_2 e^{-\kappa x} \for x<-a\\
    \psi_2 (x) &= \beta_1 e^{i k x} + \beta_2 e^{-i k x} \for -a\le x\le a\\
    \psi_3 (x) &= \gamma_1 e^{\kappa x} + \gamma_2 e^{-\kappa x} \for a<x\\
\end{align*}\tight
\begin{align*}
    k&= \sqrt{\frac{2m}{\hbar^2}E} & \kappa&= \sqrt{\frac{2m}{\hbar^2}(V_0-E)}
\end{align*}
Benutzen Sie zunächst die Bedingung, dass die Wellenfunktion für ein physikalisches Problem endlich bleiben muss. 
Wenden Sie anschließend die aus der Vorleseung bekannten Stetigkeitsbedingungen an.

\dottedlinett

Die Symmetrie des Problems impliziert, dass \(P(x)= P(-x) \implies \abs{\psi(x)}^2=\abs{\psi(-x)}^2\) und somit 
\(\psi(x) = \pm\psi(-x)\). 
Außerdem müssen für Konvergenz bei \(x\to \pm\infty\) die Konstanten \(\alpha_2=\gamma_1=0\) gewählt werden.

\begin{align*}
    \psi_1 (x) &= \alpha_1 e^{\kappa x}\\
    \psi_2 (x) &= \beta_1 e^{i k x} + \beta_2 e^{-i k x}\\
    \psi_3 (x) &= \gamma_2 e^{-\kappa x}
\end{align*}
\begin{align*}
    \psi_1(-a) &= \psi_2(-a)\\
    \alpha_1 e^{-\kappa a} &= \beta_1 e^{- i k a} + \beta_2 e^{i k a}\\
    \\
    \psi_3(a) &= \psi_2(a)\\
    \gamma_2 e^{-\kappa a} &= \beta_1 e^{i k a} + \beta_2 e^{-i k a}\\
    \\
    \pp{\psi_1}x\eval_{x=-a} &= \pp{\psi_{2}}x\eval_{x=-a}\\
    \kappa \alpha_1 e^{-\kappa a} &= ik \beta_1 e^{- i k a} - ik \beta_2 e^{i k a}\\
    \\
    \pp{\psi_3}x\eval_{x=a} &= \pp{\psi_{2}}x\eval_{x=a}\\
    -\kappa \gamma_2 e^{-\kappa a} &= ik \beta_1 e^{i k a} - ik \beta_2 e^{-i k a}\\
\end{align*}

Für \(\psi(x)=\psi(-x)\):
\begin{align*}
    \implies \beta:=\beta_1&=\beta_2\tand \alpha:=\alpha_1= \gamma_2\\
    \\
    \alpha e^{-\kappa a} &= \beta e^{i k a} + \beta e^{-i k a}\\
    -\kappa \alpha e^{-\kappa a} &= ik \beta e^{i k a} - ik \beta e^{- i k a}\\
    \\
    \alpha e^{-\kappa a} &= 2\beta \cos ka\\
    \kappa \alpha e^{-\kappa a} &= 2k\beta  \sin ka\\
    \\
    \kappa &= k \tan ka\implies \te{diskrete Energieniveaus}\\
\end{align*}

Für \(\psi(x)=-\psi(-x)\):
\begin{align*}
    \implies \beta:=\beta_1&=-\beta_2\tand \alpha:=\alpha_1=-\gamma_2\\
    \\
    -\alpha e^{-\kappa a} &= \beta e^{i k a} - \beta e^{-i k a}\\
    \kappa \alpha e^{-\kappa a} &= ik \beta e^{i k a} + ik \beta e^{- i k a}\\
    \\
    \alpha e^{-\kappa a} &= 2i\beta \sin ka\\
    -\kappa \alpha e^{-\kappa a} &= 2ik\beta \cos ka\\
    \\
    \kappa &= -\frac k {\tan ka}\implies \te{diskrete Energieniveaus}\\
\end{align*}

Die zu jedem Energiezustand zugehörige Wellenfunktion soll normalisiert sein.

Für \(\psi(x)=\psi(-x)\):
\begin{align*}
    1 &= \int_{-\infty}^{-a}\dx\psi_1\bar\psi_1 + \int_{-a}^{a}\dx\psi_2\bar\psi_2 + \int_{a}^{\infty}\dx\psi_3\bar\psi_3\\
    &= \int_{-\infty}^{-a}\dx \alpha\bar\alpha \,e^{2\kappa x} + \int_{-a}^{a}\dx2\beta\bar\beta \cos^2 kx
    + \int_{a}^{\infty}\dx  \alpha\bar\alpha \,e^{-2\kappa x}\\
    &= 2 \alpha\bar\alpha \int_{a}^{\infty}\dx e^{-2\kappa x} + 2\beta\bar\beta\int_{-a}^{a}\dx\frac{1+ \cos 2kx}{2}\\
    &= \frac{\abs\alpha}{\kappa}e^{-2\kappa a} + \abs \beta\hug{2a +\frac{\sin 2ka}{k}}\\
    &= \frac{2\cos ka\,e^{\kappa a}\abs{\beta}}{\kappa}e^{-2\kappa a} + \abs \beta\hug{2a +\frac{\sin 2ka}{k}}\\
    \abs \beta &= \hug{\frac{2\cos ka}{\kappa}e^{-\kappa a} + 2a +\frac{\sin 2ka}{k}}\inv\\
    \alpha &= 2\beta \cos ka\, e^{\kappa a}\\
\end{align*}

Für den asymmetrischen Fall \(\psi(x)=-\psi(-x)\) wechseln sich \(\cos\) und \(\sin\):
\begin{align*}
    \abs \beta &= \hug{\frac{2\sin ka}{\kappa}e^{-\kappa a} + 2a +\frac{\cos 2ka}{k}}\inv\\
    \alpha &= 2i \beta \sin ka\, e^{\kappa a}\\
\end{align*}

Hat man ein Teilchen, dessen Energie entweder \(\kappa = k \tan ka\) oder \(\kappa = -\frac k {\tan ka}\) erfüllt,
ist somit die zugehörige Wellengleichung bis auf einen komplexen Vorfaktor bestimmt, der jedoch keinen Einfluss
auf beobachtbare Größen hat. \(\beta\) kann daher der Einfachheit halber immer als \(\beta\in\R\) gewählt werden.

\section{}
In den Bereichen \(I_1 = (-\infty,-a), \ I_2 = (-a,-a)\) und \(I_3 = (a, \infty)\) ist das Potenzial konstant,
das Teilchen bewegt sich daher auf jedem jeweils wie eine freies Teilchen:
\begin{align*}
    E\psi(x) &= -\frac{\hbar ^2}{2m} \Delta \psi (x) + V\psi(x)\\
    0 &= \pp{^2\psi}{x^2} + \frac{2m}{\hbar ^2}(E-V) \psi\\
    \implies \psi(x) &= A_1 e^{i k x } + A_2 e^{-i k x } \note k = \sqrt{2m(E-V)/\hbar ^2}
\end{align*}
Damit sind die Wellenfunktionen für die drei Abschnitte jeweils der Form
\begin{align*}
    \psi_1 &= A_1 e^{i k x } + A_2 e^{-i k x }&
    \psi_2 &= B_1 e^{i k' x } + B_2 e^{-i k' x}&
    \psi_3 &= C_1 e^{i k x } + C_2 e^{-i k x }
\end{align*}
mit 
\begin{align*}
    k = \sqrt{2m E_0 /\hbar^2} \tand
    k' = \sqrt{2m (E_0 -V_0) /\hbar^2}
\end{align*}
Der Term mit \(C_2\) in \(\psi_3\) repräsentiert ein Teilchen, dass von \(x=\infty\) nach links reist; Dies 
ist per Konstruktion nicht möglich, der Vorfaktor \(C_2\) muss daher gleich null sein.\\[1ex]
Die Wellenfunktion und ihre Ableitungen, müssen an den Sprungstellen des Potenzials stetig sein:
\begin{align*}
    \te{I.:}\ \,\psi_1(-a) &= \psi_2(-a) & \te{II.:}\ \ \,\psi_2(a) &= \psi_3(a)\\
    \te{III.:}\ \pp{\psi_1}x\eval_{-a} &= \pp{\psi_2}x\eval_{-a} & \te{IV.:}\ \pp{\psi_2}x\eval_{a} &= \pp{\psi_3}x\eval_{a}
\end{align*}
\begin{align*}
    \te{I.} &\implies &A_1 e^{-i k a} + A_2 e^{i k a} 
    &= B_1 e^{- i k' a} + B_2 e^{i k' a}
    \\
    \te{II.} &\implies& B_1 e^{i k' a} + B_2 e^{-i k' a} 
    &= C_1 e^{i k a}
    \\ 
    \te{III.} &\implies& i k A_1 e^{-i k a} - i k A_2 e^{i k a} 
    &= i k' B_1 e^{- i k' a} - i k' B_2 e^{i k' a}
    \\
    \te{IV.} &\implies& i k' B_1 e^{i k' a} - i k' B_2 e^{-i k' a} 
    &= i k C_1 e^{i k a} 
\end{align*}
Sei 
\begin{align*}
    R &=\frac{A_2 \bar A_2}{A_1 \bar A_1}:=r\bar r \note[reflektierter Anteil]\\
    T &=\frac{C_2\bar C_2}{A_1 \bar A_1} :=t\bar t\note[transmittierter Anteil]\\
    1 &= R+T
\end{align*}
Damit gibt es insgesamt sieben Unbekannte und sieben Gleichung - das Gleichungssystem ist 
theoretisch lösbar, aber ich investier meine Zeit heute lieber noch in was anderes :)

\end{document}