\documentclass[ex,minted]{exercise_4.0}

\deadline{05.06.2024}

\begin{document}

\section{Herleitung der Plankschen Strahlungsformel für einen Schwarzkörper}
{\it Das Strahlungsfeld in einem thermischen Hohlraum kann als Summe stehender Wellen dargestellt werden (Moden). Aufgrund der hohen Anzahl an Moden führt man eine spektrale Modendichte ein: \(n_\nu (\nu) = \frac{8\pi \nu^2}{c^3}\).}

\subsection{\it Erläutern Sie kurz zur spektralen Modendichte:}
\begin{adjustwidth}{20pt}{}
    \subsubsection{\it Was ist spektrale Modendichte?}\vspace{2pt}
    \begin{adjustwidth}{20pt}{}
        Die spektrale Modendichte $\d n/\d\nu = \frac{8\pi\nu^2}{c_0^3}$ ist
        die Zahl der möglichen Eigenschwingungen pro Volumen eines Hohlraumresonators innerhalb des Frequenzintervalls \(\nu\) bis \(\nu+\Delta \nu\) mit \(\Delta \nu=1\u{Hz}\).
    \end{adjustwidth}
    
    \subsubsection{\it Warum wird bei der Herleitung der Formel nur über eine \(\frac18\) Kugel im \(k-\)Raum integriert und nicht über eine Vollkugel?}\vspace{2pt}
    Weil negative \(k_i\) Komponenten nicht zu neuen Lösungen führen würden. Tatsächlich unterscheiden sich die Lösungen der einzelnen E-Feld-Komponenten mit \((\pm k_x,\pm k_y,\pm k_z)\) maximal um ein Vorzeichen von der Lösung mit \((k_x,k_y,k_z)\). Dieses Vorzeichen kann einfach in den Vorfaktor der Lösung absorbiert werden, sodass alle einzigartigen Lösungen mit positiven \(k_i\) abgedeckt sind. 
    
    \subsubsection{\it Warum kommt im Ergebnis \(8\pi\) und nicht \(4\pi\) vor?}\vspace{2pt}
    \begin{adjustwidth}{20pt}{}
        Es kommt ein zusätzlicher Faktor von zwei dazu, um zu berüclsichtigen, dass jede stehende Welle eine
        beliebige Polarisationsrichtung haben kann, die man immer als Linearkombination aus zwei zueinander
        senkrecht polarisierten Wellen darstellen kann.
    \end{adjustwidth}
    
\end{adjustwidth}

\subsection{\it Zeigen sie, dass die Planksche Strahlungsformel:
\begin{align*}
    u_\nu = n_\nu (\nu) \cdot \tug{E(\nu,T)} = \frac{8\pi \nu^3 h}{c^3}\frac{1}{e^{\frac{h\nu}{k_B T}}-1},
\end{align*}
wobei \(\tug{E(\nu,T)}\) die mittlere thermische Energie einer Mode als Funktion der Temperatur \(T\) ist, gilt.\\
Tipp: Zusätzlich zu den Angaben im Skript, müssen Sie die mittlere Energie über den Mittelwert der Besetzungswahrscheinlichkeit berechnen. Die mittlere Energie (das mit den Besetzungswahrscheinlichkeiten \(P_n\) gewichtete Mittel der Zustandsenergien \(E_n\)) lässt sich durch geometrische Reihen ausdrücken:
\begin{align*}
    \tug{E(\nu,T)} &= \sum_n P_n E_n = \frac{-\dd{}\beta \sum_n e^{-\beta E_n}}{\sum_n e^{-\beta E_n}}
\end{align*}
mit \(h=2\pi\hbar\) und \(\beta= \frac{1}{k_B T}\).
}

\dottedlinete

\begin{align*}    
    \tug{E(\nu,T)} &= \sum_n P_n E_n \\
    &= \frac{-\dd{}\beta \sum_n e^{-\beta h \nu n}}{\sum_n e^{-\beta h \nu n}}\\
    &= \frac{-\dd{}\beta \sum_n \alpha^n}{\sum_n \alpha^n}\with\begin{cases}
        \alpha = e^{-\beta h \nu }<1\\
        \d\alpha = -h\nu e^{-\beta h \nu } \d\beta
    \end{cases}\\
    &= - \dd{}\beta \hug{\frac{1}{1-\alpha}}\hug{1-\alpha}\\
    &=  h\nu e^{-\beta h \nu}\dd{}\alpha \hug{\frac{1}{1-\alpha}}\hug{1-\alpha}\\
    &= h\nu e^{-\beta h \nu}\frac{1}{(1-\alpha)^2}\hug{1-\alpha}\\
    &= h\nu \frac{e^{-\beta h \nu}}{1-e^{-\beta h \nu}}\\
    &= h\nu \frac{1}{e^{\frac{h \nu}{k_B T}}-1}\\
    \\
    u_\nu &= n_\nu(\nu)\cdot \tug{E(\nu, T)} \\
    &=   \frac{8\pi \nu^3 h}{c^3}\frac{1}{e^{\frac{h\nu}{k_B T}}-1}
\end{align*}

\section{Lithium}
{\it Die Energie des tiefsten Zustands \(2s\) im Li-Atom ist \(E=-5.39 \u{eV}\), die in 20s ist \(-0.034\u{eV}\). Wir groß ist die effektive Kernladung \(Z\sub{eff}\) bzw. die Abschirmungskorrektur (Quantendefekt) \(\delta_{nl}\) und der mittlere Bahnradius des dritten Elektron in beiden Zuständen?}

\dottedlinett

Durch die Abschirmung des Kerns wird die effektive Kernladung verringert. Die Proportionalität der Energie wechselt von 
\begin{align*}
    E_n\propto \frac{1}{n^2}\qte{zu} E_{n,l} \propto \frac{1}{n^2\hug{1-\frac{\delta_{n,l}}{n}}^2},
\end{align*}
mit der Abschirmungskorrektur \(\delta_{nl}\). Es ergeben sich für die \(\delta_{nl}\):

\begin{align*}
    E_{nl} &= \frac{-13.6\u{eV}}{(n-\delta_{nl})^2}\\
    \implies \delta_{nl} &= n - \sqrt{\frac{-13.6\u{eV}}{E_{nl}}}\\
    \\
    \delta_{2s}&\approx 0.412\\
    \delta_{20s}&\approx 0.0\\
\end{align*}

Korrigiert man statt der Quantenzahl \(n\) die Kernladung \(Z\) zu einer effektiven Kernladung \(Z\sub{eff}\) ergeben sich die Werte:
\begin{align*}
    -\frac{13.6\u{eV}}{(n-\delta_{nl})^2} &= -\frac{13.6\u{eV}Z\sub{eff}^2}{n^2}\\
    \implies Z\sub{eff} &= \frac{1}{1-\frac{\delta_{nl}}{n}}\\
    \\
    Z\sub{eff}(2s) &= 1.26\\
    Z\sub{eff}(20s) &= 1.0
\end{align*}

\section{Belegung der Orbitale}
{\it 
Schreiben Sie ein Python-Skript oder Jupyter Notebook, das die Orbitalbelegungen von Mehrelektronen nach dem Madelungschema bestimmt. Wählen Sie als Eingabe die Ordnungszahl $Z$ und als Ausgabe die
s,p,d,f Nomenklatur für Unterschalen und 1,2,3,... für die Hauptschalen im Format als Dreierblock für jede Unterschale: $nlN$ mit der Hauptschale $n$, der Nebenschale $l$ und der Anzahl der Elektronen $N$ in der
Unterschale,\\
Beispiel: Argon, $Z$ = 18, Ausgabe: \py{1s2 2s2 2p6 3s2 3p6}\\
Das Programm sollte von Wasserstoff ($Z$ = 1) bis Krypton ($Z$ = 102) funktionieren und für elektrisch neutrale Atome gelten.\\\\
Überprüfen Sie das Programm mit 
\begin{enumerate}
    \item[a)] Sauerstoff
    \item[b)] Kupfer
    \item[c)] Krypton
    \item[d)] Uran
\end{enumerate}
}

\dottedlinett

\inputpy{}{7.py}

\section{Positronium}
{\it Ein Elektron kann mit seinem Antiteilchen, dem Positron (Masse $m_e$, elektrische Ladung $+e$) ein wasserstoffähnliches Atom, das Positronium, bilden, wobei Elektron und Positron unter der Wirkung ihrer gegenseitigen elektrostatischen Anziehung um den gemeinsamen Massenschwerpunkt ($S$) „kreisen“.}

\subsection{\it Berechnen Sie allgemein mit Hilfe der Bohrschen Quantenbedingung für den Gesamtbahndrehimpuls die
möglichen Bahnradien und Energiezustände des Positronium!}

\dottedlinett

Die Bohrsche Quantenbedingung für den Gesamtdrehimpuls ist
\begin{align*}
    L_n &= n\hbar
    = \mu  v_n r_N\\
\end{align*}
Im Bohrschen Atommodel muss außerdem gelten:
\begin{align*}
    0 &= F_C + F_Z\\
    &= - \frac{e^2}{4\pi \varepsilon_0 r^2} + \mu  \frac{v^2}{r}\\
    v &= \frac{e}{\sqrt{4\pi \varepsilon_0 r m _e}}\\
\end{align*}
Kombination der beiden Zusammenhänge liefert: 
\begin{align*}
    n\hbar &= e \sqrt{\frac{\mu  r_n }{4\pi\varepsilon_0 }}\\
    r_n &= \frac{4\pi \varepsilon_0\hbar^2}{\mu  e^2} n^2\\
\end{align*}
Daraus lässt sich noch die Energie berechnen als:
\begin{align*}
    E_n &= T + V\\
    &= \frac12 mv_n^2 - \frac{e^2}{4\pi\varepsilon_0 r_n}\\
    &= \frac12 m \frac{n^2\hbar^2}{\mu ^2 r_n^2} - \frac{e^2}{4\pi\varepsilon_0 r_n}\\
    &= \frac12 m \frac{n^2\hbar^2}{\mu ^2}\frac{\mu ^2 e^4}{16\pi^2 \varepsilon_0^2\hbar^4 n^4} - \frac{e^2}{4\pi\varepsilon_0} \frac{\mu  e^2}{4\pi \varepsilon_0\hbar^2 n^2} \\
    &= \frac{\mu  e^4}{32\pi^2 \varepsilon_0^2\hbar^2 n^2} - \frac{\mu  e^4}{16\pi^2 \varepsilon_0^2\hbar^2 n^2} \\
    &= -\frac{\mu  e^4}{32\pi^2 \varepsilon_0^2\hbar^2 n^2}\\
    &= -\frac{m_e^2}{m_e + m_e}\frac{e^4}{32\pi^2 \varepsilon_0^2\hbar^2 n^2}\\
    &= -\frac{m_e e^4}{64\pi^2 \varepsilon_0^2\hbar^2 n^2}\\
\end{align*}

\subsection{\it Wie groß ist die Grundzustandsenergie?}

\dottedlinett

Ein Vergleich mit der Formel für die Energieniveaus des Wasserstoff zeigt, dass der Grundzustand genau die halbe Energie des Wasserstoff hat, also \(-6.8\u{eV}\).


\subsection{\it Wie groß ist der kleinste Durchmesser?}

\dottedlinett

Genau doppelt wie das Wasserstoffatom, d.h. \(1.06 \A\).

\end{document}