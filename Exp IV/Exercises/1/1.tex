\documentclass[ex,minted,hatbasis]{exercise_4.0}

\deadline{17.04.2024}

\begin{document}

\section{Massenspektrometer}
\textit{Aus einem Strahl aus einfach ionisierten Uran-Atomen mit natürlichem Isotopengemisch \\$(0.72 \,\%\ \schemie{^{235}U}$ und
$99.28 \,\%\ \schemie{^{238}U}$) soll in einem Thomsonschen Massenspektrometer nach der Parabelmethode \schemie{^{235}U} abgeschieden
werden. Der Durchmesser des Strahls sei $1 \u{mm}$, die erreichbare Stromstärke im Strahl beträgt $0.2 \u {mA}$. Im $4\u{cm}$
langen Spektrometer herrscht ein homogenes elektrisches Feld von $5000 \ufrac Vm$ und ein homogenes Magnetfeld
von $0.01 \u T$. Wir betrachten Teilchen \schemie{^{235}U} und \schemie{^{238}U}, die sich mit gleicher Geschwindigkeit $v$ durch das
Spektrometer bewegen.}

\subsection{}
\textit{Bei welcher Geschwindigkeit der Uran-Atome überlappen sich die Strahlen der beiden Isotope direkt
hinter dem Spektrometers gerade nicht mehr?}\vspace{2ex}

Sei das Massenspektrometer nach Thomson so aufgebaut, dass der magnetische Nordpol und der elektrische negative Pol in $\e_y$-Richtung liegen, und die Atome Richtung $\e_z$ fliegen. 
Beim durchqueren des Massenspektrometer wirken auf die Uran Isotope die Coloumb-/ und Lorentzkraft: 
\begin{align*}
    \v F_L &= q \,\v v \times \v B = qv B \e_x\\ 
    \v F_C &= q \v E = q E \e_y
\end{align*}
Da jede Kraft nur in jeweils eine Richtung wirkt, kann die Bewegung des Uran für beide Richtungen einzeln betrachtet werden:
\begin{align*}
    F_L &= m a_x \implies \ddot x = \frac{q v B}{m} \implies x(t) = \frac{q v B}{2 m}t^2\\
    F_C &= m a_y \implies \ddot y = \frac{q E}{m}\implies y(t) = \frac{qE}{2 m}t^2
\end{align*}
Nimmt man an, dass die Geschwindigkeit der Teilchen entlang der $z$-Achse konstant ist gilt für den Zeitpunkt an dem die Teilchen auf den Schirm auftreffen $t_0=L/v$ mit $L$ als Distanz bis zum Schirm, und somit für die Distanzen $x_0,y_0$:
\begin{align*}
    x_0 = x(t_0) = \frac{q B L^2}{2 m v}\\
    y_0 = y(t_0) = \frac{q E L^2}{2 m v^2}
\end{align*}
\newcommand{\urana}{\schemie{^{235}U}}
\newcommand{\uranb}{\schemie{^{238}U}}

 Die Strahlen der beiden Uran-Isotope überlappen sich dann gerade nicht mehr, wenn der Anstand ihrer zentralen Strahlen $\Delta$ auf dem Schirm mindestens so groß wie der Strahldurchmesser $d$ sein. Werte die sich auf \urana{} beziehen sind im folgendem nicht gestrichen, Werte die sich auf \uranb{} beziehen sind gestrichen.

\begin{align*}
    d &= \Delta\\
    d^2 &= \hug{x_0 - x_0'}^2 + \hug{y_0 - y_0'}^2 \\
    &= \hug{\frac{q B L^2}{2 m v} - \frac{q B L^2}{2 m' v}}^2 + \hug{\frac{q E L^2}{2 m v^2} - \frac{q E L^2}{2 m' v^2}}^2 \\
    &= \frac{q^2 B^2 L^4}{4 v^2}\hug{\frac1m - \frac1{m'}}^2 + \frac{q^2 E^2 L^4}{4 v^4}\hug{\frac{1}{m} - \frac1{m'}}^2 \\
    &= \frac{q^2 L^4}{4 v^4}\hug{\frac{1}{m} - \frac1{m'}}^2\hug{v^2B^2 + E^2} \\
    0 &= v^4 - \frac{q^2 L^4}{4d^2}\hug{\frac{1}{m} - \frac1{m'}}^2\hug{v^2B^2 + E^2} \\
    &= v^4 - \frac{\eta^2 M^2}{4}\hug{v^2B^2 + E^2} \note \begin{cases}
        \eta = \frac{q L^2}{d}\\
        M = \frac{1}{m} - \frac1{m'}
    \end{cases}\\
    \implies v^2 &= \frac{\eta^2 M^2 B^2}{8} + \sqrt{\hug{\frac{\eta^2 M^2 B^2}{8}}^2 + \frac{\eta^2 M^2 E^2}{4}}\\
    &= \frac{\eta^2 M^2 B^2}{8} + \frac{\eta M}{2} \sqrt{\frac{\eta^2 M^2}{16}B^4  + E^2}\\
    \implies v &= 57307 \ufrac ms\note \frac vc= 0.191\permil
\end{align*}
Wobei verwendet wurde, dass die Atommasse für \urana{} $235\u u$ ist, und für \uranb{} $238\u u$.



\subsection{}
\textit{Wie lange dauert das Abscheiden von $1 \u{mg}$ \schemie{^{235}U}?}\vspace{2ex}

Seien $j_q$ der Teilchenstrom in $\ufrac 1s$ und $j_C$ der Ladungsstrom in $\ufrac Cs$.
\begin{align*}
    m\sub{ges} &= m N = m\, j_q\, t = \frac{M\sub{mol}\, j_C\, t}{n_A\, q} \\
    t &=\frac{m\sub{ges}\,n_A\, q}{M\sub{mol}\, j_C}\\
    &\approx\frac{10^{-3}\u{kg} \cdot \cnA \cdot \ce}{0.235\ufrac{kg}{mol}\cdot0.2\cdot10^{-3}\u A}\\
    &\approx 2.05\E6 \u s = 23.7 \u d
\end{align*}

\section{Röntgenbeugung an Kristallen}
\subsection{}
{\it
Wie groß sind Radius und Volumen von {\normalfont Ar}-Atomen in einem kalten {\normalfont Ar}-Kristall (dichteste Kugelpackung in kubisch-flächenzentriertem Gitter), wenn bei Bragg-Reflexion von Röntgenstrahlung $(\lambda = 0.45 \u{nm})$, die unter dem Winkel $\theta$ gegen die Netzebene parallel zu den Würfelflächen einfällt, das erste Reflexionsminimum bei $\theta = 43^\circ$ auftritt?\\[1ex]
{Hinweis:} Betrachtet werden Reflexionen an der Ebene parallel zu den Würfelflächen. Deswegen ist, wenn $a$ die Länge der kubischen Elementarzelle ist, die Distanz zwischen zwei Streuebenen ist $d = a/2$ (Siehe Abbildung 1).}\vspace{2ex}

Für ein Reflexionsminimum muss die Wegdifferenz zweier Strahlen die an den zwei Streuebenen reflektiert werden $n\lambda-\lambda/2 \note n\in\N$ betragen (destruktive Interferenz).

\begin{align*}
    \lambda(n - 1/2) &\peq \Delta s = 2 d \sin\theta = a \sin\theta \\
    a &= \frac{\lambda}{2\sin\theta} \for n=1\\
    &= \frac{0.45\u{nm}}{\sin(43^\circ)}\\
    &\approx 6.60 \A 
\end{align*}
Aus der Geometrie folgt für den Radius $r$:
\begin{align*}
    a^2 &= (2r)^2 + (2r)^2\\ 
    \implies r &= \frac{a}{\sqrt 8}\\
    &\approx 2.33 \A\\
    \\
    V &= \frac 43 \pi r^3\\
    &\approx 53.0\A^3
\end{align*}

\subsection{}
\textit{In der Vorlesung wurde ein {\normalfont NaCl}-Kristall (kubisch-flächenzentriertes Gitter) in Braggscher Anordnung mit Röntgenlicht einer Molybdän-Anode vermessen. Die Bragg Winkel ergaben sich zu $7.22^\circ$ und $6.41^\circ$.
Berechnen Sie den Abstand der Streuebenen des {\normalfont NaCl}-Kristalls. Bestimmen Sie mit Ihrem Ergebnis die
Avogadro-Zahl. $(\rho\sub{NaCl} = 2.163 \ufrac g{cm^3})$\\[1ex]
{Hinweis:} Die Molybdän Anode emittiert ein kontinuierliches Röntgenspektrum mit zwei charakteristischen Linien bei $E_{K_\alpha} = 17.4 \u{keV}$ und $E_{K_\beta} = 19.6 \u{keV}$. Die molaren Massen von {\normalfont Na} und {\normalfont Cl} sind Ihnen
bekannt.}\vspace{2ex}

Die den charakteristischen Linien zugehörige Wellenlängen sind:
\begin{align*}
    \lambda &= \frac cf = \frac{hc}{E}\\
    \\
    \lambda_{K_\alpha} &\approx 71.3\u{pm}\\
    \lambda_{K_\beta} &\approx 63.3\u{pm}
\end{align*}

Für die Glanzwinkel gilt die Bragg-Bedingung:
\begin{align*}
    n\lambda &= 2d \sin\theta\\
    d &= \frac{n\lambda}{2\sin\theta}\\
    \\
    d_{K_\alpha} &\approx 2.83\A\\
    d_{K_\beta} &\approx 2.83\A
\end{align*}
Für die Glanzwinkel der beiden charakteristischen Linien kommt der gleiche Abstand der Streuebenen von $2.83\A$ heraus. \\

In der Elementarzelle des \schemie{NaCl} befinden sich vier Natriumatome und vier Chloratome. Die Kantenlänge der Elementarzelle ist $a=2d$ (siehe Nr.2 (a)). Damit lässt sich die Avogadro-Konstante wie folgt bestimmen:

\begin{align*}
    \rho\sub{NaCl} &= \frac{4m\sub{Na} + 4m\sub{Cl}}{a^3} \\
    &= 4\frac{M\sub{Na} + M\sub{Cl}}{n_A a^3} \\
    n_A &= 4\frac{M\sub{Na} + M\sub{Cl}}{\rho\sub{NaCl} \,a^3} \\
    &\approx 5.96\E{23} \ufrac1{mol}
\end{align*}

Der Literaturwert für die Avogadro-Konstante ist $\cnA$, damit ist der relative Fehler des experimentellen Wertes etwa \(1.02\%\).

\section{Tintenstrahldrucker}
{\it Die Düse eines Tintenstrahldruckers spritzt Tröpfchen auf ein Blatt Papier. Ein Tintentröpfchen hat die Masse $m = 1.3\E{-10}\u{kg}$ und eine Ladung von $Q = -1.5 \E{-13} C$. Die Tröpfchen treten orthogonal in ein homogenes E-Feld der Länge $L = 1.6 \u{cm}$ mit einer Geschwindigkeit von $v = 20 \ufrac ms$ ein.}

\subsection{}
{\it Wie stark muss das E-Feld gewählt werden, damit die Tröpfchen beim Verlassen des E-Feldes $0.5 \u{mm}$ vertikal abgelenkt werden?}\vspace{2ex}

Sei $d$ die Distanz, die das Tröpfchen nach dem Kondensator bis zum Papier zurücklegen muss. Im Kondensator wirkt auf das Tröpfchen wirkt die Coloumbkraft:
\begin{align*}
    m a &= F_C = q E\\
    \ddot x &= \frac{qE}{m}\\
    \implies x(t) &= \frac{qE}{2m} t^2
\end{align*}
Bei konstanter Geschwindigkeit Richtung Papier ($v_z = \const$) benötigt das Tröpfchen $t_0 = \frac{L}{v}$ um aus dem Kondensator wieder raus zu kommen.
Es hat zu diesem Zeitpunkt die x-Koordinate und die Geschwindigkeit:
\begin{align*}
    x_0 &= x(t_0) = \frac{q E L^2}{2 m v^2}\\
    v_0 &= \dot x(t_0) = \frac{q E L}{m v}\\
\end{align*}
Zwischen Papier und Kondensator wirken keine Kräfte, somit ist die finale x-Koordinate des Tröpfchens:
\begin{align*}
    x' &= x_0 + v_0 t'\\ 
     &= \frac{q E L^2}{2 m v^2} + \frac{q E L d}{m v^2}\\
     &= \frac{q E L \hug{L + 2d}}{2m v^2}\\
     \implies E &= \frac{2m v^2 x'}{q L^2} \note d=0\\
     &\approx -1.35 \E6\ufrac Vm
\end{align*}


\subsection{}
{\it Die Tropfen haben eine Größe von $0.5 \u{mm}$ auf dem Papier. Wie weit entfernt müsste das Papier mindestens sein, damit zwei Tropfen mit einer um eins unterschiedlichen Elementarladung $Q$ und $Q + e$ sich nicht überlappen?}\vspace{2ex}
\begin{align*}
    D &\peq x_2' - x_1' \\
    &= \frac{E L (q_2 - q_1)\hug{L + 2d}}{2m v^2}\\
    &= -\frac{e E L\hug{L + 2d}}{2m v^2}\\
    d &= -\frac{m v^2 D}{e E L} - \frac L2\\
    &\approx 7500 \u m
\end{align*}

\section{Einheiten umrechnen}
\textit{Schreiben Sie ein kurzes Programm oder Jupyter Notebook, um Einheiten umrechnen zu können. Erstellen
Sie eine Tabelle mit den Werten 1-10 des Ausgabewertes und berechnen Sie den entsprechenden Zielwert. Rechnen sie folgende Einheiten um: Joule in Elektronvolt, Meter in Ångström, kg in Stoffmenge (Mol) für \schemie{H2O}, 
Elektronvolt in Wellenlänge und Frequenz (für Photonen).}\vspace{1ex}

\inputpy{Python Code}{1.py}

\end{document}