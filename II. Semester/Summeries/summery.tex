\title{Experimental Physics II}

\documentclass[11pt,letterpaper]{article}
\textwidth 6.5in
\textheight 9.in
\oddsidemargin 0in
\headheight 0in
\setlength\parindent{0pt}

\usepackage[summery, de]{custom_2.0}
\newcommand\alt{\overset{(a.)}{=}}
\begin{document}
\maketitle
  
\tableofcontents

\section{Thermodynamik}
\subsection{Haubtsätze}

 \subsubsection{I. Haubtsatz}
 {
    Die gesamt Energie ist in einem geschlossenen System zeitlich konstant.
    \begin{equation*}
        \Delta U = \Delta Q + \Delta W
    \end{equation*} 
    $\Delta U =$ die Änderung der (gesamten) ineren Enrgie eines geschlossenen Systemes\\
    $\Delta Q =$ von außen zugeführte Wärmeenergie\\
    $\Delta W =$ vo außen zugeführte mechanische Energie\\ 
 }

 \subsubsection{II. Haubtsatz}
 {
    Wärme fließt von selbst immer nur vom wärmeren zum kälteren Körper, nicht
    umgekehrt.\\
    In einem abgeschlossenen System nimmt die Entropie nicht ab $\Delta S \ge09$.
 }

 \subsubsection{III. Haubtsatz}
 {
    Es ist prinzipiell nicht möglich, den absoluten Temperaturnullpunkt (T = 0 K)
    zu erreichen.
 }

\subsection{Wärmetransport}
{
    \subsubsection{Diffusion}
    {
        Netto-Teilchenstromdichte bei Diffusion:
        \begin{equation*}
            \vec{j} = -D\cdot \vec{\nabla}n\\
        \end{equation*}
    }
    \subsubsection{Konduktion}
    {
        Wärmestromdichte bei Konduktion:
        \begin{align*}
            \vec{j}_Q &= -\lambda\cdot \vec{\nabla}T\\
            \frac{\abs{\d Q}}{S\cdot \dt} &= \lambda \frac{\abs{\Delta T}}{d}
        \end{align*}
    }
    \subsubsection{Wärmestrahlung}
    {
        Gesamtstrahlungsleitung (nach Stefan-Boltzmann-Gesetz):
        \begin{equation*}
            P = \varepsilon \sigma A T^4\\
        \end{equation*}
    }
}

\subsection{Zustandsänderungen}
{
    \subsubsection{Isotherm}
    {
        \begin{align*}
            \Delta T = 0
        \end{align*}
    }
    \subsubsection{Isobar}
    {
        \begin{align*}
            \Delta p &= 0\\
        \end{align*}
    }
    \subsubsection{Isochor}
    {
        \begin{align*}
            \Delta V &= 0 \\
        \end{align*}
    }
    \subsubsection{Adiabatisch}
    {
        \begin{align*}
            \Delta Q &= 0\\
        \end{align*}
        \begin{align*}
            \Delta U &= \frac{f}{2} N k \Delta T = -p \Delta V = -\frac{N k T }{V} \Delta V\\
            \frac{f}{2} \frac{\Delta T}{T} &= - \frac{\Delta V }{V}\\
             T V ^{\kappa -1} &=  \text{const}\\
            p V ^{\kappa } &=  \text{const}\\
            p^{1-\kappa} T ^{\kappa} &=  \text{const}\\
        \end{align*}
    }
}

\subsection{Schallgeschwindikeit}
{
    Für niedrige Frequenzen:
    \begin{align*}
        v_s &= \sqrt{\frac{p}{\rho}}
    \end{align*}
    FÜr hohe Frequenzen
    \begin{align*}
        v_s &= \sqrt{\kappa \frac{p}{\rho}}
    \end{align*}
}

\subsection{Wärmekraftmaschine / Carnot - Kreisprozess}
{
    \begin{align*}
    \end{align*}
}

\subsection{Energien}
{
    Im Gas:
    \begin{align*}
        \braket{E_{tot}} &= \frac{f}{2}\mathrm{k}_B T
    \end{align*}
    Allgemein:
    \begin{align*}
        \Delta  E &= c M\Delta T
    \end{align*}

}

\subsection{Entropie}{
    \ \ \ \ Klassischer, thermischer Entropiebegriff:
    \begin{align*}
        \d S &= \frac{\d Q_{rev}}{T} \\
        \Delta S &= \int_K \frac{\d Q_{rev}}{T} 
    \end{align*}
    Statistischer Entropiebegriff:

    Die Entropie ist ein Maß für die Wahrscheinlichkeit eines Zustandes, d.h. für
    die Anzahl der mikroskopischen Realisierungsmöglichkeiten eines vorgegebe-
    nen makroskopischen Zustandes.
    \begin{align*}
        S &= k_B \ln n_{RM}
    \end{align*}
    \begin{align*}
        n_{RM} \hat{=} \text{ mikroskopische Realisierungsmöglichkeiten für einen makroskopischen Zustand}
    \end{align*}
}

\subsection{Thermodynamik realer Gase und Flüssigkeiten}
{
    \subsubsection{Clausius-Clapeyron-Gleichung}
    {
        \begin{align*}
            Q(T) &= T\cdot \frac{\d p_S}{\d T} \cdot (V_G - V_L)
        \end{align*}

        $Q$ ist die Verdampfungswärme für eine vorgegebene Stoffmenge (z.B. ein Mol), $V_G$
        bzw. $V_L$ sind die entsprechenden Volumina falls sich die Substanz vollständig in
        der Gasphase bzw. im flüssigen Aggregatzustand befindet (Liquid), für die gleiche
        Stoffmenge. Entsprechendes gilt für die anderen Phasenübergänge.\\

        Mit den Approximationen $V_G \ll V_L$ und $Q = \mathrm{const}$ (d.h. nicht T-abhängig) kann
        man aus der Clausius-Clapeyron-Beziehung die Abhängigkeit des Dampfdruckes
        von der Temperatur näherungsweise berechnen:

        \begin{align*}
            W \frac{\d T}{T} = \d p_S \cdot V_G
        \end{align*}
        Integration führt nun zu folgendem Ausdruck: 
        \begin{align*}
            p_S &= p_S^0 \cdot e ^{-\frac{Q}{Nk}\hug{\frac{1}{T}-\frac{1}{T_0}}}
        \end{align*}
    }
    \subsubsection{Zustandsgleichung des realen Gases}
    {
        \begin{equation*}
            n R T = \left(p+\frac{a n^2}{V^2}\right) \cdot(V-n b)\\\
        \end{equation*}
    }
    \subsubsection{Boltzmann-Faktor}
    {
        \begin{align*}
            N(R) \propto  e^{-\frac{E}{kT}}
        \end{align*}
    }
}

\subsection{Zeichen und ihre Bedeutung}
\begin{align*}
    p &\Longleftrightarrow \text{Druck/Pressure}\\
    V &\Longleftrightarrow \text{Volumen}\\
    T &\Longleftrightarrow \text{Temperatur}\\
    f &\Longleftrightarrow \text{Zahl der Freiheitsgrade}\\
    n &\Longleftrightarrow \text{Stoffmenge (in mol)}\\
    N &\Longleftrightarrow \text{Stoffmenge}\\
    U &\Longleftrightarrow \text{innere Energie}\\
    Q &\Longleftrightarrow \text{Wärmeenergie}\\
    \vec{j} &\Longleftrightarrow \text{Netto-Teilchenstromdichte bei Diffusion} = -D\cdot \vec{\nabla}n\\
    \d R &\Longleftrightarrow \text{Reduzierte Wärmemenge} \ = \frac{\d Q}{T}\\
    \d S &\Longleftrightarrow \text{Entropie} \ = \frac{\d Q_{rev}}{T}\\
    \kappa &\Longleftrightarrow \text{Adiabatenindex} \ = \frac{c_P}{c_V} &= \frac{f+2}{f}=1+\frac{2}{f}\\
    \\
    N_A &\Longleftrightarrow \text{Avogadro-Konstante}\\
    R &\Longleftrightarrow \text{allgemeine Gaskonstante}\\
    k &\Longleftrightarrow \text{Boltzmann-Konstante}\\
    c &\Longleftrightarrow \text{spezifische Wärmekapazität}\ = \frac{\Delta Q}{M\Delta T}\\
    D &\Longleftrightarrow \text{Diffusionskonstante}\\
    \sigma &\Longleftrightarrow \text{Stefan-Boltzmann-Konstante}\ = 5.77\cdot 10^{-8}\ufrac{W}{m^2K^4}\\
    \varepsilon &\Longleftrightarrow \text{Absorptionsgrad}\ \le 1\\\\
\end{align*}

\section{Elektrostatik}
\subsection{Haubtsätze}
\subsubsection{Gaußsches Gesetz}
\begin{align*}
    \vnabla \cdot \vec E &= \frac \rho {\epsilon_0}\\
    \implies Q &= \oiiint_{V} \rho \dV = \epsilon_0\oiint_{S} \vec E \cdot \dS\\
    \implies \vec E &= \frac{1}{4\pi \epsilon_0 \epsilon_r} \frac{Q}{r^2}\e_r\\
\end{align*}

\subsection{Kondensator}

\subsubsection{Plattenkondensator}
Elektrisches Feld:
\begin{align*}
    E = \frac{Q}{\epsilon_0\epsilon_r A}
\end{align*}
Spannung:
\begin{align*}
    U &= \frac Qd \overset{reihe}{\implies}U =U_1+U_2 \overset{parallel}{\implies} U_1=U_2
\end{align*}
Kapazität:
\begin{align*}
    C = \frac{Q}{U} = \epsilon_0\epsilon_r \frac{A}{d} 
    \overset{reihe}{\implies} \frac{1}{C_{ges}}= \frac{1}{C_1} + \frac{1}{C_2} + \dots + \frac{1}{C_n}
    \overset{parallel}{\implies} C=C_1 + C_2 + \dots + C_n
\end{align*}

\subsubsection{Zylinderkondensator}
Elektrisches Feld:
\begin{align*}
    E = \frac{Q}{2\pi r l \epsilon_0\epsilon_r}
\end{align*}
Kapazität:
\begin{align*}
    C = 2\pi \epsilon_0\epsilon_r \frac{l}{\ln\hug{\frac{R_2}{R_1}}} 
\end{align*}

\subsubsection{Kugelkondensator}
Elektrisches Feld:
\begin{align*}
    E = \frac{Q}{4\pi r^2 \epsilon_0\epsilon_r}
\end{align*}
Kapazität:
\begin{align*}
    C = 4\pi \epsilon_0\epsilon_r\hug{\frac{1}{R_1} - \frac{1}{R_2}}^{-1} 
\end{align*}

\section{Elektrik}
\subsection{Aufgabenformate}
\begin{enumerate}
    \item Driftgeschwindigkeit bestimmen
    \begin{align*}
        v_D &\alt \sigma_{el} E \alt \frac{E}{\varrho} \alt  a\cdot \tau \alt \frac{j}{n q} \\
        &= \frac{I}{n q A}
    \end{align*}
    \item Mittlere Flugzeit zwischen Kollisionen von Elektronen im Leiter
    \begin{align*}
        \tau &= \frac{v_D}{a} = \frac{v_D m_e}{qE} 
        =\frac{I m_e}{n A q^2 E} 
    \end{align*}
    \item Mittlerer Weglänge:
    \begin{align*}
        \Lambda &=  v \tau\\
        \frac{1}{2}m_e v^2 &= \frac{3}{2}k_B T\\
        v &= \sqrt{\frac{3k_B T}{m_e }}\\
        \Lambda &= \tau \sqrt{\frac{3k_B T}{m_e }}
    \end{align*}
\end{enumerate}

\subsection{Strom}
\begin{align*}
    I &= \deriv{Q}{t} = \oint j \dA = - \deriv{}{t}\int \varrho_{el} \dV\\
\end{align*}

\subsection{Beweglichkeit}
\begin{align*}
    \vec v_D &= \mu \cdot \vec E = \frac{I}{n q}\\ % n 
    n &= \frac{Q_{frei}}{V}
    \mu &= \frac{ q }{m } \tau_s % mu = beweglichkeit
\end{align*}

\subsubsection{Kontinuitätsgleichung}
\begin{align*}
    \vnabla j(r,t) &= - \partiald{}{t} \varrho_{el}(r,t)
\end{align*}

\subsection{Knotenregel}
\begin{align*}
    0 &= \sum_i I_i\\
\end{align*}

\subsection{Maschenregel}
\begin{align*}
    0 &= \sum_i U_i\\
    -U_0 &= \sum_{i\ge1} U_i
\end{align*}
 
\subsection{Zeichen und ihre Bedeutung}
\begin{align*}
    I = \frac{U}{R}&\Longleftrightarrow \text{Stromstärke}\\
    U = I R&\Longleftrightarrow \text{Spannung}\\
    R = \frac{I}{U} = \rho_R \frac{L}{A}&\Longleftrightarrow \text{Widerstand}\\
    \vec j = \frac{I}{A} = \sigma_e \cdot \vec E = n q \vec v_D &\Longleftrightarrow \text{Stromdichte}\\
    n = \frac{N}{V}&\Longleftrightarrow \text{Ladungsträgerdichte}\\
    \vec v_D = \mu \vec E = \vec a \tau_s&\Longleftrightarrow \text{Driftgeschwindigkeit}\\
    \tau_s = \frac{\mu m }{q}&\Longleftrightarrow \text{mittlere Zeit zwischen zwei Stößen}\\
    \mu = \frac{q}{m} \tau_s&\Longleftrightarrow \text{Beweglichkeit }\\
    \sigma_e = \frac{q^2}{m}n \tau_s&\Longleftrightarrow \text{elektrische Leitfähigkeit}\\
    \rho_R = \frac{1}{\sigma_e}&\Longleftrightarrow \text{spezifischer Widerstand}\\
    &\Longleftrightarrow \text{}\\
    &\Longleftrightarrow \text{}\\
    &\Longleftrightarrow \text{}\\
\end{align*}

\section{Magnetostatik}
\subsection{4. Maxwell'sche Gleichung}
Differenzialform:
\begin{align*}
    \vnabla \times \vec B &= \mu_0 \vec j+ \mu_0\epsilon_0 \partiald{\vec E }{t}
\end{align*}
Integralform:
\begin{align*}
    \oint_S \vec B \dS &= \mu_0 I
\end{align*}
Biot-Savart-Gesetz:
\begin{align*}
    \d \vec B = - \frac{\mu_0 I}{4\pi} \cdot \frac{\vec r \times \dS}{r^3}
\end{align*}

 \end{document}