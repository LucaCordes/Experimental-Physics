 %%%%% Layout Variables %%%%%
 \newcommand{\ConstTitle}{Experimental Physics II}

 \documentclass[11pt,letterpaper]{article}
 \textwidth 6.5in
 \textheight 9.in
 \oddsidemargin 0in
 \headheight 0in
 
 \usepackage[book,dark]{custom_0.1}
 
 \begin{document}{
 
 \maketitle
   
 \tableofcontents

\section{Thermodynamik}{
 \subsection{I. Haubtsatz}
 {
    Die gesamt Energie ist in einem geschlossenen System zeitlich konstant.
    \begin{equation*}
        \Delta U = \Delta Q + \Delta W
    \end{equation*} 
    $\Delta U =$ die Änderung der (gesamten) ineren Enrgie eines geschlossenen Systemes\\
    $\Delta Q =$ von außen zugeführte Wärmeenergie\\
    $\Delta W =$ vo außen zugeführte mechanische Energie\\ 
 }

 \subsection{II. Haubtsatz}
 {
    Wärme fließt von selbst immer nur vom wärmeren zum kälteren Körper, nicht
    umgekehrt.\\
    In einem abgeschlossenen System5 nimmt die Entropie nicht ab
 }

 \subsection{III. Haubtsatz}
 {
    Es ist prinzipiell nicht möglich, den absoluten Temperaturnullpunkt (T = 0 K)
    zu erreichen.
 }

\subsection{Wärmetransport}
{
    \subsubsection{Diffusion}
    {
        Netto-Teilchenstromdichte bei Diffusion:
        \begin{equation*}
            \vec{j} = -D\cdot \vec{\nabla}n\\
        \end{equation*}
    }
    \subsubsection{Konduktion}
    {
        Wärmestromdichte bei Konduktion:
        \begin{align*}
            \vec{j}_Q &= -\lambda\cdot \vec{\nabla}T\\
            \frac{\abs{\di Q}}{S\cdot \dt} &= \lambda \frac{\abs{\Delta T}}{d}
        \end{align*}
    }
    \subsubsection{Wärmestrahlung}
    {
        Gesamtstrahlungsleitung (nach Stefan-Boltzmann-Gesetz):
        \begin{equation*}
            P = \varepsilon \sigma A T^4\\
        \end{equation*}
    }
}

 \subsection{Formeln}{

    Zustandsgleichung des idealen Gases:
    \begin{equation*}
        p\cdot V = N\cdot k\cdot T
    \end{equation*}

    Gesamte kinetische Energie, abhängig von Freiheitsgraden:
    \begin{equation*}
        E_{kin}^{tot} = \frac{f}{2} k T
    \end{equation*}

    Expansionsarbeit:
    \begin{equation*}
        \abs{\Delta W} = p\Delta V
    \end{equation*}

    \begin{align*}
        \langle E_{kin}^{atom}\rangle&=\frac{3}{2}k\cdot T \\
        \frac{\Delta L}{L}&=\alpha \Delta T\\
        \frac{\Delta V}{V} &= 3 \alpha \Delta T = \gamma \Delta T\\
    \end{align*}

    Zeichen und ihre Bedeutung:
    \begin{align*}
        p &\Longleftrightarrow \text{Druck/Pressure}\\
        V &\Longleftrightarrow \text{Volumen}\\
        T &\Longleftrightarrow \text{Temperatur}\\
        f &\Longleftrightarrow \text{Zahl der Freiheitsgrade}\\
        n &\Longleftrightarrow \text{Stoffmenge (in mol)}\\
        N &\Longleftrightarrow \text{Stoffmenge}\\
        U &\Longleftrightarrow \text{innere Energie}\\
        Q &\Longleftrightarrow \text{Wärmeenergie}\\
        \vec{j} &\Longleftrightarrow \text{Netto-Teilchenstromdichte bei Diffusion} = -D\cdot \vec{\nabla}n\\
        \di R &\Longleftrightarrow \text{Reduzierte Wärmemenge} \ = \frac{\di Q}{T}\\
        \di S &\Longleftrightarrow \text{Entropie} \ = \frac{\di Q_{rev}}{T}\\
        &\Longleftrightarrow \text{}\\
        &\Longleftrightarrow \text{}\\
        &\Longleftrightarrow \text{}\\
        &\Longleftrightarrow \text{}\\
        \\
        N_A &\Longleftrightarrow \text{Avogadro-Konstante}\\
        R &\Longleftrightarrow \text{allgemeine Gaskonstante}\\
        k &\Longleftrightarrow \text{Boltzmann-Konstante}\\
        c &\Longleftrightarrow \text{spezifische Wärmekapazität}\ = \frac{\Delta Q}{M\Delta T}\\
        D &\Longleftrightarrow \text{Diffusionskonstante}\\
        \sigma &\Longleftrightarrow \text{Stefan-Boltzmann-Konstante}\ = 5.77\cdot 10^{-8}\ufrac{W}{m^2K^4}\\
        \varepsilon &\Longleftrightarrow \text{Absorptionsgrad}\ \le 1\\
        &\Longleftrightarrow \text{}\\
    \end{align*}
 }
 }}
 
 \end{document}