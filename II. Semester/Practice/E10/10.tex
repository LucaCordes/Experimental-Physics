\newcommand{\deadline}{23.06.2023}

\documentclass[11pt]{article}

\usepackage[exercise,ex]{custom_2.1}

\begin{document}

\section{Barlow'sches Rad}
\subsection{}
\begin{adjustwidth}{20pt}{}
    Das Barlow'sche Rad besteht aus einem leitendem Rad, welches drehbar 
    gelagert ist und auf einer Seite in eine leitende Flüssigkeit eingetaucht 
    ist. Legt man nun eine Spannung zwischen Achse und Flüssigkeit an,
    fließt ein Strom radial durch das Rad. Wird außerdem das Rad
    mit einem B-Feld parallel zur Achse durchsetzt,
    fängt das Rad "von alleine" an zu drehen, da die Elektronen welche im Rad fließen,
    durch die Lorentzkraft seitwärts abgelenkt werden, und einen Teil
    dieser tangentialen Bewegung durch Reibung an die Atomrümpfen im Rad weitergeben.
\end{adjustwidth}
    
\subsection{}
\begin{adjustwidth}{20pt}{}
    Die Rotationsgeschwindigkeit wird wie in (a) erläutert durch die Interaktion zwischen 
    den Elektronen und den Atomen erzeugt, folglich kann die Rotationsgeschwindigkeit bei gleichem Strom erhöht werden, 
    indem der Widerstand im Rad erhöht wird, bzw. der Widerstand der Flüssigkeit
    verringert wird. (Oder trivialer Weise helfen auch: die Stromstärke erhöhen,
    den mechanischen Widerstand in der Lagerung verringern)
\end{adjustwidth}

\section{Teilchen im Magnetfeld}
\subsection{}
\begin{align*}
    F_{el} &= F_Z\\
    \frac{1}{4\pi \epsilon_0}\frac{q Q}{r^2} &= m \frac{v^2}{r}\\
    v^2 &=  \frac{1}{4\pi \epsilon_0}\frac{q Q}{m r}\\
    \omega_0^2 &= \frac{1}{4\pi \epsilon_0}\frac{q Q}{m r^3}\\
\end{align*}

\subsection{}
\begin{align*}
    0 &= F_{Z} + F_{el} + F_{L}\\
    0 &= m \frac{v^2}{r} -\frac{1}{4\pi \epsilon_0}\frac{q Q}{r^2} - v q B \\
    0 &= v^2 - \frac{v q B}{m} -\frac{1}{4\pi \epsilon_0}\frac{q Q}{m r} \\
    0 &= \omega^2 - \omega\frac{ q B}{m r^2} -\frac{1}{4\pi \epsilon_0}\frac{q Q}{m r^3} \\
    \omega(B) &= \underbrace{\frac{ q B}{2 m r^2}}_{\gamma} \pm\sqrt{\frac{1}{4} \hug{\frac{ q B}{2 m r^2}}^2 + \underbrace{\frac{1}{4\pi \epsilon_0}\frac{q Q}{m r^3}}_{\omega_0}} \\
    &= \gamma \pm\sqrt{\frac{\gamma^2}{4} + \omega_0} \\
    \omega(0) &=  \omega_0\\
    \partiald{\omega}{B}(0) &= \frac\gamma B = \frac{ q}{2 m r^2}\\
    \partiald{^2\omega}{B^2}(0) &= \pm\frac{1}{\sqrt{\omega_0}}\\
    \omega(B) &\approx \omega_0 + \frac{ q}{2 m r^2} B
     \ \ \bhug{\pm\frac{1}{2\sqrt{\omega_0}}B^2 }
\end{align*}
\begin{adjustwidth}{20pt}{}
    \emph{Beobachtungen:} Die Kreisfrequenz \(\omega\) ist proportional 
    zum überlagertem B-Feld, d.h. \(\omega \propto B\). Interpretiert man den
    Aufbau als ein Elektron, dass um einen Atomkern kreist, so würde dies bedeuten,
    dass das Durchdringen von Materie mit einem B-Feld Energie benötigt oder Energie generiert (umgewandelt) wird, 
    da das Elektron an kinetischer Energie gewinnt/verliert.
    Eine interessante Folge ist, dass die kinetische Energie ein Maximum bei einer bestimmten 
    Raumrichtung der Winkelgeschwindigkeit hat. Es ist somit möglich, dass sich die Elektronen bei einer
    mit einem Magnetfeld durchsetzten Materie ordnen, d.h. ihre Winkelgeschwindigkeit parallel zueinander wird,
    da dies einem Minimum im Potenzial entspricht. Das durch die bewegte Ladung 
    erzeugte B-Feld überlagert sich daher nun konstruktiv; d.h. es könnte bei genug 
    geordneten Atomen ein messbares Magentfeld entstehen, welches sogar relativ stabil wäre,
    da für die "Entparallelisierung" der Winkelgeschwindigkeiten eine Arbeit geleistet 
    werden muss. Das Resultat ähneld einem Permanentmagneten.
\end{adjustwidth}

\section{Helmholtzspulen}
\subsection{}
\begin{adjustwidth}{20pt}{}
    Für eine Spule:
\end{adjustwidth}
\begin{align*} 
    \d \vec B &= - \frac{\mu_0 I}{4\pi} \cdot \frac{\vec r \times \ds}{r^3}\\
    \vec B  &= - \frac{\mu_0 I}{4\pi} \oint_{\mathcal L } \frac{\vec r \times \ds}{r^3}\\
    &= - \frac{\mu_0 I}{4\pi} \oint_{\mathcal L} \frac{\hug{z\e_z+R \e_r+ \theta \e_\theta} \times (\e_\theta \dtheta)}{\hug{z^2 + R^2}^{\frac 32}}\\
    &= - \frac{\mu_0 I}{4\pi} \oint_{\mathcal L} \dtheta \frac{z\e_r-R \e_z}{\hug{z^2 + R^2}^{\frac 32}}\\
    &= - \frac{\mu_0 I}{4\pi} \frac{1}{\hug{z^2 + R^2}^{\frac 32}}\oint_{0}^{2\pi} \dtheta R\hug{z\e_r-R \e_z}\\
    &= \frac{\mu_0 I}{4\pi} \frac{1}{\hug{z^2 + R^2}^{\frac 32}}\Biggl(2\pi R^2 \e_z - z R{\oint_{0}^{2\pi} \dtheta \e_r}\Biggr)\\
    &= \frac{\mu_0 I}{4\pi} \frac{1}{\hug{z^2 + R^2}^{\frac 32}}\Biggl(2\pi R^2 \e_z - \underbrace{z R\oint_{0}^{2\pi} \dtheta \hug{\cos\theta,\sin\theta,0}^T}_{=0}\Biggr)\\
    &= \frac{\mu_0 I}{2} \frac{R^2 }{\hug{z^2 + R^2}^{\frac 32}} \e_z 
\end{align*}
\begin{adjustwidth}{20pt}{}
    Das Prinzip der Superposition ergibt nun für Überlagerung von zwei 
    gleichgroßen Spulen:
\end{adjustwidth}
\begin{align*}
    \vec B &= \vec B_1 + \vec B_2\\
    &= \frac{\mu_0 I}{2} R^2 \hug{{\hug{(z+d/2)^2 + R^2}^{-\frac 32}}
    + {\hug{(z-d/2)^2 + R^2}^{-\frac 32}} } \e_z\\
\end{align*}

\subsection{}
\begin{align*}
    \vec B(z) &= \frac{\mu_0 I}{2} R^2 \hug{{\hug{(z+d/2)^2 + R^2}^{-\frac 32}}
    + {\hug{(z-d/2)^2 + R^2}^{-\frac 32}} } \e_z\\
    \vec B(0) &= \mu_0 I R^2 {\hug{(d/2)^2 + R^2}^{-\frac 32}}\e_z\\
    \partiald{\vec B}{z}(0) &= 0 \since[Achsensymmetrisch]\\
    \partiald{^2\vec B}{z^2}(0) &= \frac{3}{2}\mu_0 I R^2  \frac{\hug{4(d/2)^2 - R^2}}{\hug{(d/2)^2+R^2}^{\frac 72}}\e_z z^2\\
    \vec B_T(z) &\approx\vec B(z) \approx  \frac{\mu_0 I R^2}{\hug{(d/2)^2 + R^2}^{\frac 32}}\e_z \hug{1+ \frac 34 \frac{\hug{4(d/2)^2 - R^2}}{\hug{(d/2)^2+R^2}^{2}} z^2}
\end{align*}

\subsection{}
\begin{align*}
    \vec B_T = \text{const}\implies 0= {4(d/2)^2 - R^2} \implies d=R
\end{align*}

\subsection{}
\begin{align*}
    \vec B_T(0) &= \frac{\mu_0 I R^2} {\hug{(d/2)^2 + R^2}^{\frac 32}}\e_z\\
    &= \frac{\mu_0 n I^* R^2} {\hug{(d/2)^2 + R^2}^{\frac 32}}\e_z\\
    &\approx \frac{\munull \cdot 1000 \cdot 1.5\u A \cdot 10^2\u{cm^2}}{\hug{(10\u{cm}/2)^2 + 10^2\u{cm^2}}^{\frac 32}}\e_z\\
    &\approx 13.5 \u{mT}\cdot \e_z
\end{align*}


\section{Hall-Effekt}
\subsection{}
\begin{adjustwidth}{20pt}{}
    Für den Hall-Effekt ist die Stärke des Stroms relevant, welche 
    der Ladung entspricht die pro Zeiteinheit eine Fläche durchströmt. 
    Man sieht nun, dass ein Strom von positven Ladungsträgern, der in die 
    entgegengesetzte Richtung relativ zu den Negativen fließt, den Gesammtstrom vergrößer würde,
    und somit auch die Hallspannung. 
\end{adjustwidth}


\subsection{}
\begin{align*}
    0 &= F_L + F_{el}\\
    &= q v B + \frac{U_H}{b}\\
    &= q v B + \frac{U_H}{b}\\
    U_H &= -dqvB\\
    \\
    j &= n q v\\
    v &= \frac{j}{nq}\\
    \\
    U_H &= -\frac{b j B}{n q}\\
    &= -\frac{I B}{d n q}\\
    &\approx -\frac{1\u A \cdot 0.1\u T}{1\u{mm}\cdot1.1\cdot 10^{29}\ufrac{1}{m^3}\cdot (\chargeelec)}\\
    &\approx 5.68\u{nV}
\end{align*}
\begin{adjustwidth}{20pt}{}
    Man könnte die Hallspannung erhöhen und damit einfacher messbar machen,
    indem man bei gleicher Stromdichte die Ladungsträgerdichte verringert, 
    da dann die Ladungsträger schneller sind und der Einfluss der Lorenzkraft
    somit größer. Realisieren ließe sich dies, indem man statt Kupfer z.B. einen
    Halbleiter verwendet.
\end{adjustwidth}

\subsection{}
\begin{align*}
    U_H &= -\frac{b j B}{n q}\\
    &= -b v B\\
    v &= -\frac{U_H}{b B}\\
    &\approx \frac{5.68\u{nV}}{5\u{mm}\cdot 0.1\u T}\\
    &\approx 11.4\ufrac{\micro m}{s}
\end{align*}

\subsection{}
\begin{align*}
    \rho &= \frac{A U}{l I}\\
    U &= \frac{\rho l I}{d b}\\
    U_{\te{Cu}} &\approx \frac{1.7\cdot 10^{-8}\,\Omega\mathrm m \cdot 6\u{mm} \cdot 0.7 \u A}{1\u{mm} \cdot 5\u{mm}}\\
    &\approx 14.3\u {\micro V}\\
    U_{\te{Si}} &\approx \frac{4.6\cdot 10^{-2}\,\Omega\mathrm m \cdot 6\u{mm} \cdot 0.7 \u A}{1\u{mm} \cdot 5\u{mm}}\\
    &\approx 38.6 \u V
\end{align*}
\begin{adjustwidth}{20pt}{}
    Die beiden spezifischen Widerstände unterscheiden sich um sechs Größenordnungen,
    somit ist eine starke Differenz der benötigten Spannung erwartet.
\end{adjustwidth}

\section{Magnetfeld eines stromdurchflossenden Leiters}

\subsection{}
\begin{align*}
    \d \vec B &= - \frac{\mu_0 I}{4\pi} \cdot \frac{\vec r \times \ds}{r^3}\\
    \vec B  &= - \frac{\mu_0 I}{4\pi} \oint_{\mathcal L } \frac{\vec r \times \ds}{r^3}\\
    &= - \frac{\mu_0 I}{4\pi} \hug{\oint_{\te{Schleife} } \frac{\vec r \times \ds}{r^3}
    + \int_{\te{gerader Leiter} } \frac{\vec r \times \ds}{r^3}}\\
    - \frac{\mu_0 I}{4\pi}\oint_{\te{Schleife} } \frac{\vec r \times \ds}{r^3} &= B_{\text{Spule}}(0)\notethat[siehe Nr.3 (a)]\\
    &= \frac{\mu_0 I}{2} \frac{R^2}{(z^2+ R^2{\frac 32})} \e_z\\
    &= \frac{\mu_0 I}{2} \frac{1}{R} \e_z
\end{align*}
\begin{align*}
    \int_{\te{Gerader Leiter}} \frac{\vec r \times \ds}{r^3} &=
    \int_{\mathcal L} \frac{(R\e_y - x\e_x)\times (\dx\e_x)}{(x^2+R^2)^{\frac32}}\\ 
    &= R\e_z\int_{\mathcal L} \frac{\dx}{(x^2+R^2)^{\frac32}}\\
    &= \frac{\e_z}{R^2}\int_{-\infty}^{\infty}\frac{\d x}{(x^2/R^2+1)^{\frac32}} \\
    &\eq \frac{\e_z}{R}\int_{-\infty}^{\infty}\frac{\cosh u}{(\sinh^2u+1)^{\frac32}} \du\\
    &\eq \frac{\e_z}{R}\int_{-\infty}^{\infty}\frac{1}{\cosh^2u} \du\with \sinh^2u+1=\cosh^2u\\
    &= \frac{\e_z}{R} \tanh u \eval_{-\infty}^{\infty}\\
    &= \frac{2}{R} \e_z\\
    \\
    \vec B_{\te{ges}}  &= \frac{\mu_0 I}{2} \frac{1}{R} \e_z - \frac{\mu_0 I}{4\pi} \frac{2}{R} \e_z\\
    &= \frac{\mu_0 I}{2} \frac{1}{R} \e_z \hug{1-\frac{1}{\pi}}\\
\end{align*}
\begin{adjustwidth}{20pt}{}
    \con Substitution: \(\sinh u=\frac{x}{R}\rightarrow\du \cosh u=\frac\dx R\)\\
\end{adjustwidth}

\subsection{}
\begin{align*}
    \int_{\te{Gerader Leiter}} \frac{\vec r \times \ds}{r^3} &= 0 \since \vec r \parallel \ds\\
    \int_{\te{Schleife*}} \frac{\vec r \times \ds}{r^3} &= 
    \int_{\te{Oben}} \frac{\vec r \times \ds}{r^3} + \int_{\te{Unten}} \frac{\vec r \times \ds}{r^3}\\
    &\eq0 \\
    \vec B_{\te{ges}}&=0
\end{align*}
\begin{adjustwidth}{20pt}{}
    \con Da nach der Rechtenhandregel der Beitrag zum B-Feld der von einem Leiterstückchen oben geleistet
    wird, von dem Leitstückchen darunter weggehoben wird, d.h. die Beträge der beiden Stücken sind gleich,
    jedoch zeigen sie in die genau entgegengesetzte Richtung zueinander.
\end{adjustwidth}


\section{Maxwell-Gleichungen}
\begin{enumerate}
    \item Maxwellgleichung:\\
    \begin{align*}
        \vnabla \cdot \vec E &= \frac{\rho}{\epsilon_0} \\
        \oint_{\mathcal V}  \vnabla \cdot \vec E \dV&= \frac{1}{\epsilon_0}\oint_{\mathcal V} \rho\dV \\
        \oint_{\mathcal S}  \vec E \dS&= \frac{Q}{\epsilon_0} 
    \end{align*}
    \begin{adjustwidth}{20pt}{}
        Diese Gleichung beschreibt, dass das elektrische Fluss durch eine
        geschlossene Oberfläche propotialnal ist zur Ladungen innerhalb diese. 
        Das macht die Ladung zu Quellen des elektrischen Feldes. 
        Außerdem impliziert dies bereits die \(\frac{1}{r^2}\) Propotionalität
        des E-Feldes da dies der Proportionalität einer Oberfläche in 3 Dimensionen bei 
        Skalierung entspricht.
    \end{adjustwidth}
    
    \item Maxwellgleichung:\\
    \begin{align*}
        \vnabla \cdot \vec B &= 0\\
        \oint_{\mathcal V}\vnabla \cdot \vec B \dV&= 0\\
        \oint_{\mathcal S} \vec B \dS &= 0
    \end{align*}
    \begin{adjustwidth}{20pt}{}
        Diese Gleichung sagt aus, dass der über eine geschlossene Oberfläche
        summierte magentische Fluss gleich null ist, d.h. es gibt 
        keine magnetische Ladung gibt.
    \end{adjustwidth}

    
    \item Maxwellgleichung:\\
    \begin{align*}
        \vnabla \times \vec E &= -\partiald{\vec B}{t}\\
        \oint_{\mathcal S} \vnabla \times \vec E \dS &= -\oint_{\mathcal S}\partiald{\vec B}{t}\dS\\
        \oint_{\mathcal L} \vec E \d \vec x &=- \oint_{\mathcal S} \partiald{\vec B}{t}\dS
    \end{align*}
    \begin{adjustwidth}{20pt}{}
        Diese Gleichung sagt aus, dass das entlang einer geschlossen Linie summierte E-Feld 
        bei einem nicht veränderndem B-Feld gleich null ist. Konkret auf 
        Schaltungen angewand entspricht dies der Knotenregel 
        (für Stromkreise mit ausschließlich
        konzentrierten Bauelementen), d.h.
        dass die Summe aller Ströme gleich null ist.
        Gibt es jedoch ein zeitlich veränderndes B-Feld, so erzeugt dieses
        ein Wirbel-E-Feld, sodass das durch eine geschlossene Linie summierte
        E-Feld nun proportional zur negativen magnetischen Flussänderung innerhalb der geschlossenen 
        Linie ist (Induktion).
    \end{adjustwidth}
    
    \item Maxwellgleichung:\\
    \begin{align*}
        \vnabla \times \vec B &= \mu_0\vec j + \mu_0\epsilon_0\partiald{\vec E}{t}\\
        \oint_{\mathcal S} \vnabla \times \vec B \dS &=  \mu_0\oint_{\mathcal S}\vec j \dS + \mu_0\epsilon_0\oint_{\mathcal S} \partiald{\vec E}{t}\dS\\
        \oint_{\mathcal L} \vec B \d \vec x &= \mu_0 I+ \mu_0 \epsilon_0\oint_{\mathcal S} \partiald{\vec E}{t}\dS
    \end{align*}
    \begin{adjustwidth}{20pt}{}
        Diese Gleichung sagt aus, dass das über eine geschlosse Linie summierte 
        B-Feld gleich dem Strom ist der zwischen der geschlossenen Linie fließt, 
        plus die zeitliche Änderung des elektrischen Flusses zwischen der Linie. 
    \end{adjustwidth}
\end{enumerate}


\end{document}