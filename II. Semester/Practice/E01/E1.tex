\newcommand{\ConstExercise}{01}
\newcommand{\ConstDeadline}{14.04.2023}

\documentclass[11pt,letterpaper]{article}
\textwidth 6.5in
\textheight 9.in
\oddsidemargin 0in
\headheight 0in

\usepackage[dark,exp]{custom_0.1}

\begin{document}

%%%%% Document %%%%%

\begin{enumerate}
    \item \textbf{Zum Aufwärmen}
    \begin{enumerate}
        \item 
            Extensive Größen verändern sich, abhängig von der Größe des betrachteten Systems.
            Beispiel: Masse, Volumen, ...\\[1ex]
            Im Gegensatz dazu sind intensive Größen, solche die unabhängig von der Größe des Systems sind.
            Beispiel: Temperatur, Massendichte, ...
        \item 
            Man kann die beiden Objekte über einen dritten Stoff, 
            welcher mit beiden verträglich ist, 
            in thermischen Kontakt miteinander bringen.
            Ist sowohl das erste Objekt als auch das zweite mit dem dritten im thermischen Gleichgewicht, 
            dann ist auch das erste Objekt mit dem zweiten im thermischen Gleichgewicht. \\
            Man müsste um diese Methode zu verwenden, darauf achten, dass der dritte Stoff mit 
            mindestens einem der anderen objekte im themischen Gleichgewicht ist, noch bevor man 
            sie thermisch miteinander verbindet. 
        \item 
            Der Miniskus senkt sich zunächst etwas ab, da das Glas in welchem die Flüssigkeit
            sich befindet, sich zuerst ausdehnt. Das Volumen des Glasgefäßes steigt, während das
            der Flüssigkeit zunächst konstant bleibt. Das Resultat ist eine kurzfristige, sehr kleine 
            Absenkung des Flüssigkeitspegels.
        \item
            Die Querschnittsfläche des Lochs vergrößert sich, da sich das Blech ausdehnt.
        \item 
            Schätzungsweise $4\Celsius \ -\ 100\Celsius $, da erst ab $4\Celsius$ gilt: $ V \propto T$
            ; Bei $100\Celsius$ würde für gewöhnlich das Wasser gasförmig werden, jedoch erhöht sich bei
            der Erhitzung in einerm festen Volumen der Druck, was dieser Entwicklung zu einem gewissen Grad entgegen wirken sollte.
    \end{enumerate}

    
    \item \textbf{Eiesenbahnrad}
    \begin{enumerate}
        \item 
            \begin{align*}
                \Delta T &= \frac{1}{\alpha} \frac{\Delta r}{r_{innen}}\\
                &= \frac{1}{13\cdot 10^{-6}\Kelvin^{-1}}\frac{0.4\m[m]}{170\m[m]}\\
                &\approx 181 \Kelvin
            \end{align*}
        \item 
            \begin{align*}
                E &=  1.8 \cdot 10^{11} \ufrac{N}{m^2} \\
                \sigma &= E \varepsilon\\
                &= 1.8 \cdot 10^{11} \ufrac{N}{m^2}\frac{-0.4\m[m]}{170.4\m[m]}\\
                &\approx -4.22\cdot 10^{8} \ufrac{N}{m^2}\\
            \end{align*}
    \end{enumerate}

    
    \item \textbf{Räumliche Ausdehnung}
    \begin{enumerate}
        \item 
            \begin{align*}
                \Delta V &= \gamma \cdot \Delta T _{rel}\\
                V' &=V\cbrace{1+ \gamma \cdot \Delta T _{rel}}\\
                &= 12000\cbrace{1+\mathbf{L} \cdot 1.05\cdot 10^{-3} \Kelvin ^{-1} \cdot 30\Kelvin}\\
                &\approx 12400 \mathrm{L}\\
            \end{align*}
        \item 
            \begin{align*}
                \rho &= \frac{m}{V}\\
                \rho& \propto \frac{1}{V}\\
                \\
                \rho' &= \rho \cdot \frac{1}{\Delta V_{rel}}\\
                 &= \rho \cdot \frac{1}{1+\gamma \Delta T}\\
                 &= 7.3\cdot 10^{3} \ufrac{kg}{m}\frac{1}{1+11\cdot 10^{-6} \Kelvin ^{-1}\cdot 1180\Kelvin}\\
                 &\approx 7.21\cdot 10^{3} \ufrac{kg}{m^3}
            \end{align*}
        \item 
            \begin{align*}
                 h &= \frac{V}{\pi r^2}\\
                 \Delta h &= \frac{\Delta V}{\pi r^2}\\
                 &= \frac{\gamma \Delta T\cdot V}{\pi r^2}\\
                 r&= \sqrt{\frac{\gamma \Delta T\cdot V}{\pi \Delta h}}\\
                 &= \sqrt{\frac{1.6\cdot 10^{-4}\Kelvin^{-1} \cdot 1\Kelvin \cdot 0.5\mathrm{cm}^3}{\pi\cdot 1\mathrm{cm}}}\\
                &\approx 5.04\cdot10^{-5} = 50.4 \,\m[\mu]\\
            \end{align*}
    \end{enumerate}

    \newpage

    \item \textbf{Bimetall II}
    \begin{enumerate}
        \item 
            \begin{align*}
                l_1 &= l(1+\alpha_1 \Delta T)\\
                l_2 &= l(1+\alpha_2 \Delta T)\\ 
                \Delta l_{rel} &= \frac{l_1}{l_2}\\
                &= \frac{1+\alpha_1 \Delta T}{1+\alpha_2\Delta T }\\
                \\
                \Delta l_{rel} &= \Delta U_{rel}\\
                &= \frac{2\pi(r+d)}{2\pi r} \\
                &= 1+ \frac{d}{r}=\frac{1+\Delta T \alpha_1}{1+\alpha_1\Delta T}\\
                r &= \frac{d}{\frac{1+\Delta T \alpha_1}{1+\alpha_2\Delta T }-1}\\  
            \end{align*}
        \item 
            \begin{align*}
                r &= \frac{d}{\frac{1+\Delta T \alpha_1}{1+\Delta T \alpha_2}-1}\\
                &= \frac{d({1+\Delta T \alpha_2})}{\Delta T (\alpha_1-\alpha_2)}\\ 
                &\approx \frac{d}{\Delta T (\alpha_1-\alpha_2)}\\
                &= \frac{10^{-3}\,\mathrm{m}}{300\Kelvin \cdot 1.85\cdot 10^{-5}\Kelvin^{-1}}\\
                &\approx 1.80 \cdot 10^{-1} \,\mathrm{m} = 18\,\mathrm{cm}\\
            \end{align*}
    \end{enumerate}

    \item \textbf{Ideale Gase}
    \begin{enumerate}
        \item Richtig (Satz von Avogadro)
        \item Falsch, da:
            \begin{align*}
                \begin{cases}
                     N_1 = N_2\\m_1=m_2
                \end{cases}\Longleftrightarrow 
                \begin{cases}
                    N_1 = N_2\\N_1\bar{m}_1=N_2 \bar{m}_2
                \end{cases}
                \Longrightarrow \bar{m}_1 = \bar{m}_2
                \Longrightarrow \text{im Allgemeinen falsch}
            \end{align*}
            
        \item Falsch, da:
            \begin{align*}
                \bar{v}_i &= \int_{0}^{\infty} v\,p_i(v)\,\di v\\
                &= \sqrt{\frac{8 k_B T}{\pi m_i}}\\
                \\
                \bar{v}_1 &= \bar{v}_2\\
                \sqrt{\frac{8 k_B T}{\pi m_1}} &= \sqrt{\frac{8 k_B T}{\pi m_2}}\\
                m_1 &= m_2 \Longrightarrow \text{im Allgemeinen falsch}\\
            \end{align*}
    \end{enumerate}

    \item \textbf{Energieerhaltung}
    \begin{align*}
        \Delta {E_{pot}} &= \Delta \braket{E_{kin}}\\
        mg\Delta h &= \frac{3}{2}k_B \Delta T\\
        \Delta T &= \frac{2}{3}\frac{mg\Delta h}{k_B}\\
         &= \frac{2}{3} \frac{18.015\cdot 10^{-3}\ufrac{kg}{mol}\cdot 1/6.022 \cdot  10^{-23}\mathrm{mol}\cdot \accelerationearth\cdot 110\meter}
         {1.380649\cdot 10^{-23} \ufrac{J}{K}}\\
         &\approx 1.56\Kelvin 
    \end{align*}
\end{enumerate}

\end{document}