\newcommand{\ConstExercise}{2}
\newcommand{\ConstDeadline}{21.04.2023}

\documentclass[11pt,letterpaper]{article}
\textwidth 6.5in
\textheight 9.in
\oddsidemargin 0in
\headheight 0in

\usepackage[dark,exp]{custom_0.1}

\begin{document}

%%%%% Document %%%%%

\begin{enumerate}
  \item \textbf{Wärmeleitung I}
    \begin{enumerate}
      \item
        \begin{align*}
          \vec{j}_Q &= -\lambda\cdot \vec{\nabla}T\\
          j_Q&= \lambda \frac{\abs{\Delta T}}{d}\\
          \\
          &= 0.8\,\ufrac{W}{mK} \frac{40\,^\circ\mathrm{K}}{5\cdot 10^{-3}\, \mathrm{m}}\\
          &= 6400\,\ufrac{W}{m^2}
        \end{align*}

      \item
        Die Temperatur der äußeren Seite der inneren Scheibe ist gleich der Innentemperatur,
        analog dazu ist die Temperatur der äußere Seite der äußeren Scheibe gleich der Außentemperatur.
        Dies liegt daran das der Temperaturverlauf stetig wird, denn die Formel sagt für einen 
        abrupter Temperaturwechsel eine unendliche Wärmestromdichte vorraus:
        \begin{align*}
          \lim_{d\to 0}j_Q &= \lim_{d\to 0}\lambda \frac{\abs{\Delta T}}{d} = \infty \quad \text{, für } \Delta T >0
        \end{align*}

      \item
        \begin{align*}
          j_Q&= \lambda \frac{\abs{\Delta T}}{d}\\
          \Delta T_{i} &= \frac{j_Q d_i}{\lambda_i}\\
          \sum_{i} \Delta T_{i} &= \abs{T_1 - T_2} = j_Q \sum_i \frac{ d_i}{\lambda_i}\\
          j_Q &= \frac{\Delta T}{ \sum_i \frac{ d_i}{\lambda_i}}\\
          \\
        \end{align*}
    \end{enumerate}

  \item \textbf{Spezifische Wärme}
    \begin{align*}
    \end{align*}


  \item \textbf{Thermische Eigenschaften von Stickstoff}
    \begin{enumerate}
      \item
        \begin{align*}
        \end{align*}

      \item
        \begin{align*}
        \end{align*}

      \item
        \begin{align*}
        \end{align*}

      \item
        \begin{align*}
        \end{align*}

      \item
        \begin{align*}
        \end{align*}
    \end{enumerate}

\newpage
  \item \textbf{Um welches Gas handelt es sich? }
    \begin{enumerate}
      \item
        \begin{align*}
        \end{align*}

      \item
        \begin{align*}
        \end{align*}
    \end{enumerate}

  \item \textbf{Steigende Luftblase}
    \begin{enumerate}
      \item
        \begin{align*}
        \end{align*}

      \item
        \begin{align*}
        \end{align*}
    \end{enumerate}

\end{enumerate}

\end{document}