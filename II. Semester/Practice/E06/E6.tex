\newcommand{\ConstExercise}{6}
\newcommand{\ConstDeadline}{18.05.2023}

\documentclass[11pt,letterpaper]{article}
\textwidth 6.5in
\textheight 9.in
\oddsidemargin 0in
\headheight 0in

\usepackage[exp]{custom_0.1}
\usepackage{physics}

\begin{document}

%%%%% Document %%%%%

\begin{enumerate}
    \item \textbf{Elektrostatisches Pendel}
        \begin{enumerate}
            \item
            In einem Kugelkondensator mit den Radien $r_1$ und 
            $r_2$ gilt, herrscht für  $r_1<r<r_2$ das Feld einer 
            im Kugelmittelpunkt sitzenden Punktladung:   
            \begin{align*}
                \vec{E}(r) &= \frac{Q}{4 \pi \varepsilon_0 r^2}\e_{r}\\
                \\
                \phi(r) &= - \int \vec{E}(r) \cdot \di \vec{r}\\
                &= \frac{Q}{4 \pi \varepsilon_0 r} + \phi_0\\
                \\
                U &= \phi(r_1) - \phi(r_2)\\
                &= \frac{Q}{4 \pi \varepsilon_0} \cbrace{\frac{1}{r_1}-\frac{1}{r_2}}\\
                \\
                C_K &= \lim_{r_2\to \infty}\frac{Q}{U}\\
                &= \lim_{r_2\to \infty}\frac{Q}{\frac{Q}{4 \pi \varepsilon_0} \cbrace{\frac{1}{r_1}-\frac{1}{r_2}}}\\
                &= 4 \pi \varepsilon_0 r_1 = 4 \pi \varepsilon_0 R \\
            \end{align*}

            \item
            \begin{align*}
                0 &= \ddot{r} - \frac{F}{m}\\
                &= \ddot{r} - \frac{E(r) q}{m}\\
                &= \ddot{r} - \frac{\frac{U}{d} \cdot U C_K}{m}\\
                &= \ddot{r} - \frac{C_K U^2}{d m}\\
                \vec{r}(t) &= \frac{C_K U^2}{2 d m}t^2 + v_0 t + r_0\\
            \end{align*}
            Die Kupferkugel sei nun auf der linken Seite des Kondensators, 
            sodass gilt $v_0=0\ \land r_0=0$. Dann ergibt sich die Periodendauer
            aufgrund des symmetrischen Schwingvorganges aus: $P = 2 t_0\ |\ r(t_0)=d $
            \begin{align*}
                \vec{r}(t_0) &= \frac{C_K U^2}{2 d m}t_0^2 = d\\
                t_0 &= \sqrt{\frac{2d^2 m}{C_K U^2}}\\
                \\
                P &= 2 t_0 = 2\sqrt{\frac{2d^2 m}{C_K U^2}} 
                = 2 \frac{d}{U}\sqrt{\frac{2m}{C_K}}
                = \frac{2}{E}\sqrt{\frac{2m}{C_K}}
            \end{align*}
        \end{enumerate}

    
    \item \textbf{Influenzmaschine}
        \begin{align*}
        \end{align*}

    
    \item \textbf{Zylinderkondensator}
        \begin{enumerate}
            \item
            \begin{align*}
            \end{align*}

            \item
            \begin{align*}
            \end{align*}

        \end{enumerate}

    
    \item \textbf{Unendlich ausgedehnte Leiterplatte}
        \begin{enumerate}
            \item
            \begin{align*}
            \end{align*}

            \item
            \begin{align*}
            \end{align*}

        \end{enumerate}

    
    \item \textbf{Potenzialdifferenz}
        \begin{enumerate}
            \item
            \begin{align*}
            \end{align*}

            \item
            \begin{align*}
            \end{align*}

            \item
            \begin{align*}
            \end{align*}

            \item
            \begin{align*}
            \end{align*}

        \end{enumerate}

    
    \item \textbf{Kondensatorauf-/ und -umladung}
        \begin{enumerate}
            \item
            \begin{align*}
            \end{align*}

            \item
            \begin{align*}
            \end{align*}
            
            \item
            \begin{align*}
            \end{align*}

            \item
            \begin{align*}
            \end{align*}

        \end{enumerate}
\end{enumerate}

\end{document}