\newcommand{\ConstExercise}{6}
\newcommand{\ConstDeadline}{19.05.2023}

\documentclass[11pt,letterpaper]{article}
\textwidth 6.5in
\textheight 9.in
\oddsidemargin 0in
\headheight 0in

\usepackage[exp]{custom_0.1}

\begin{document}

%%%%% Document %%%%%

\begin{enumerate}
    \item \textbf{Elektrostatisches Pendel}
        \begin{enumerate}
            \item
            In einem Kugelkondensator mit den Radien $r_1$ und 
            $r_2$ gilt, herrscht für  $r_1<r<r_2$ das Feld einer
            im Kugelmittelpunkt sitzenden Punktladung:   
            \begin{align*}
                \vec{E}(r) &= \frac{Q}{4 \pi \varepsilon_0 \varepsilon_r r^2}\e_{r}\\
                \\
                \phi(r) &= - \int \vec{E}(r) \cdot \di \vec{r}\\
                &= \frac{Q}{4 \pi \varepsilon_0 \varepsilon_r r} + \phi_0\\
                \\
                U &= \phi(r_1) - \phi(r_2)\\
                &= \frac{Q}{4 \pi \varepsilon_0 \varepsilon_r} \cbrace{\frac{1}{r_1}-\frac{1}{r_2}}\\
                \\
                C_K &= \lim_{r_2\to \infty}\frac{Q}{U}\\
                &= \lim_{r_2\to \infty}\frac{Q}{\frac{Q}{4 \pi \varepsilon_0 \varepsilon_r} \cbrace{\frac{1}{r_1}-\frac{1}{r_2}}}\\
                &= 4 \pi \varepsilon_0 \varepsilon_r r_1 \approx 4 \pi \varepsilon_0 R \\
            \end{align*}

            \item
            \begin{align*}
                0 &= \ddot{r} - \frac{F}{m}\\
                &= \ddot{r} - \frac{E(r) q}{m}\\
                &= \ddot{r} - \frac{\frac{U}{d} \cdot U C_K}{m}\\
                &= \ddot{r} - \frac{C_K U^2}{d m}\\
                \vec{r}(t) &= \frac{C_K U^2}{2 d m}t^2 + v_0 t + r_0\\
            \end{align*}
            Die Kupferkugel sei nun auf der linken Seite des Kondensators, 
            sodass gilt $v_0=0\ \land r_0=0$. Dann ergibt sich die Periodendauer
            aufgrund des symmetrischen Schwingvorganges aus: \\$P = 2 t_0\ |\ r(t_0)=d $
            \begin{align*}
                \vec{r}(t_0) &= \frac{C_K U^2}{2 d m}t_0^2 = d\\
                t_0 &= \sqrt{\frac{2d^2 m}{C_K U^2}}\\
                \\
                P &= 2 t_0 = 2\sqrt{\frac{2d^2 m}{C_K U^2}} 
                = 2 \frac{d}{U}\sqrt{\frac{2m}{C_K}}
                = \frac{2}{E}\sqrt{\frac{2m}{C_K}}\\
            \end{align*}
        \end{enumerate}

    
    \item \textbf{Influenzmaschine}
        \begin{enumerate}
            \item[] Aufbau:
                Der wichtigste Bestandteil einer Influenzmaschine ist das 
                Rad, auf welchem sich innen und außen jeweils voneinander isolierte
                Leiterplatten befinden, die sich in entgegengesetzte Richtung voneinander drehen können. 
                Die beiden Seiten der Leiterplatten werden
                mit zwei gegenüberliegenden, leitenden Bürsten abgestrichen und so jeweils 
                mit einem der beiden Polen eines Kondensators verbunden. Des weiteren 
                wird einmal die Außenseite und einmal die Innenseite, jeweils an 
                gegenübergelegenden Seiten abgestrichen und leitend verbunden. 
            \item[] Skizze:
            \item[] Funktionsweise:
            \begin{enumerate}
                \item Es sei angenommen, dass es eine Leiter-Platte gibt, die 
                rein zufällig eine beliebig kleine Ladung ungleich null hat.
                Diese Asymmetrie wird durch die Influenzmaschine 
                ausgenutzt und verstärkt.
                \item 
                Die geladene Leiterplatte verursacht bei vorbeirotierenden 
                Platten durch den Effekt der Influenz eine Ladungstrennung.
                \item 
                Berührt nun einer dieser durch Influenz geladenen Platten eine der Bürsten, die die Außen-/
                Innenseite miteinander verbinden, fließen Elektronen von einer Platte zur andern 
                , aus einer geladenen Platte sind jetzt mehrere geworden.
                \item
                Dreht sich das Rad nun weiter verstärkten sich die Ladungsdifferenzen,
                jedesmal wenn zwei Seiten elektrisch verbunden werden und mindestens eine
                der Platten geladen ist. Die Spannung zwischen den Platten schaukelt sich hoch.
                \item 
                Hin und wieder kommen die Platten auch in Kontakt mit den Polen des 
                Kondensators, sodass ein Ladungsausgleich stattfindet und der 
                Kondensator sich langsam auflädt.
            \end{enumerate}
        \end{enumerate}

    
    \item \textbf{Zylinderkondensator}
        \begin{enumerate}
            \item 
            \begin{align*}
                \frac{\rho}{\varepsilon_0} &= \vnabla \cdot\vec{E}\\
                \int_K\frac{\rho}{\varepsilon_0} \,\di^3r &= \int _K \vnabla \cdot\vec{E} \,\di^3r\\
                \frac{Q}{\varepsilon_0} &= \oint_A \vec{E} \cdot \di \vec{A}\\
                &= \oint_A E\cdot \dA\\
                &= \int_0^l \int_{0}^{2\pi}r E \, \di \theta \di r\\
                &= 2\pi lrE\\
                E &= \frac{Q}{2\pi \varepsilon_0 lr}
            \end{align*}
            \begin{align*}
                U &= \int_{R_1}^{R_2} E \,\dr\\
                &= \int_{R_1}^{R_2} \frac{Q}{2\pi \varepsilon_0 lr} \,\dr\\
                &= \frac{Q}{2\pi \varepsilon_0 l}\ln\cbrace{\frac{R_2}{R_1}}\\
                \\
                C_Z &= \frac{Q}{U}\\
                &= \frac{Q}{\frac{Q}{2\pi \varepsilon_0 l}\ln\cbrace{\frac{R_2}{R_1}}}\\
                &= \frac{2\pi \varepsilon_0 l}{ \ln\cbrace{\frac{R_2}{R_1}}}\\
            \end{align*}

            \item
            \begin{align*} 
                C_Z &= \frac{2\pi \varepsilon_0 l}{ \ln\cbrace{\frac{R_2}{R_1}}}\\
                & \approx \frac{2\pi \cdot \cepsilon \cdot 1.5\,\mathrm{m}}{ \ln\cbrace{\frac{6\,\mathrm{mm}}{2\,\mathrm{mm}}}}\\
                & \approx 75.9\,\mathrm{nF}
            \end{align*}

        \end{enumerate}

    
    \item \textbf{Unendlich ausgedehnte Leiterplatte}
        \begin{enumerate}
            \item
            Aus der Symmetrie folgt, dass die Richtung des E-Feldes überall 
            der normalen Vektor $\e_n$ der Ebene ist, welcher von der Ebene wegzeigt.
            Sei $E=\sabs{\vec{E}}\ \land \dS=\sabs{\dS} \ \land \ \hat{E}= \frac{\vec{E}}{E}  \ \land \ \hat{\di S}= \frac{\dS}{\di S}$.
            \begin{align*}
                \oint_A \vec{E} \cdot \di \vec{A}\ &=\frac{Q}{\varepsilon_0}\\
                \oint_A \hat{E}\cdot \hat{\di S} \cdot E \,\di A\ &=\frac{\sigma A}{\varepsilon_0}\\
                \oint_A E \, \di A &= \frac{\sigma A}{\varepsilon_0}\\
                2 E A &=\frac{\sigma A}{\varepsilon_0}\\
                E &=\frac{\sigma }{2 \varepsilon_0}\\
                \vec{E} &= \frac{\sigma }{2 \varepsilon_0}\e_n\\
                &= \begin{cases}
                    \frac{\sigma }{2 \varepsilon_0} \e_z \quad \text{, für }z>0\\
                    -\frac{\sigma }{2 \varepsilon_0} \e_z \quad \text{, für }z<0
                \end{cases}
            \end{align*}

            \item
            \begin{align*}
                E &= \frac{\sigma }{2 \varepsilon_0}\\
                &\approx \frac{1\,\ufrac{C}{m^2}}{2\cdot \cepsilon}\\
                &\approx 5.65\cdot 10^{10} \ufrac{V}{m}
            \end{align*}

        \end{enumerate}

    
    \item \textbf{Potenzialdifferenz}
        \begin{enumerate}
            \item
            Höheres Potenzial im Vergleich zu was? Je nach Referenzpunkt 
            kann das Potenzial der Platte A oder B gleich null sein, oder jeden anderen 
            Wert annehmen, nur die Differenz des Potenzials von A und B ist fest als 
            $750 \,\mathrm{V}$ gegeben. Dementsprechend hängt die Antwort dieser
            Frage vom Betrachter ab, und ist im Rahmen dieser Aufgabe nicht klar beantwortbar.

            \item
            \begin{align*}
                E &= \frac{U}{d}\\
                &= \frac{750\,\mathrm{V}}{0.1\,\mathrm{m}}\\
                &= 7500 \ufrac{V}{m}
            \end{align*}

            \item
            \begin{align*}
                W &= U e\\
                &= 750 \,\mathrm{eV}\\
                &\approx 1.20 \cdot 10^{-16} \,\mathrm{J}
            \end{align*}

            \item
            \begin{align*}
                T &= W\\
                v &= \sqrt{\frac{2W}{m}}\\
                &\approx \sqrt{\frac{2\cdot 750 \,\mathrm{eV}}{9.11\cdot 10^{-31}\,\mathrm{kg}}}\\
                &\approx 1.62\cdot 10^{7}\,\ufrac{m}{s}
            \end{align*}

        \end{enumerate}

    
    \item \textbf{Kondensatorauf-/ und -umladung}
        \begin{enumerate}
            \item
            \begin{align*}
                I_{Q} &= \frac{U_Q}{R} = \derivative{Q}{t} = C \derivative{U_Q}{t}\\
                0 &=  \derivative{U_Q}{t} -\frac{U_Q}{C\,R}\\
                U_Q(t) &= U_Q(0)\cdot e^{-\frac{t}{C\,R}}\\
            \end{align*}
            \begin{align*}
                U_C(t) &= U_Q(0) - U_Q(t)\\
                &= U_Q(0)\cbrace{1- e^{-\frac{t}{C\,R}}}\\
                \\
                I_{C}(t) &= C \derivative{U_C(t)}{t}\\
                &= -\frac{U_Q(0)}{R}e^{-\frac{t}{C\,R}}
            \end{align*}

            \item
            \begin{align*}
                W &= \int_{0}^{Q} U\,\di q\\
                &= \int_{0}^{Q} \frac{q}{C}\,\di q\\
                &= \frac{1}{2}\frac{Q^2}{C}\\
                &= \frac{1}{2}C\,U^2
            \end{align*}
            
            \item
            Abgeklemmte Spannungsquelle:
            \begin{align*}
                W_1 &= W_1' + W_2'\\
                W_1' &= W_2'\\ 
                \Longrightarrow W_1' &= W_2' = \frac{W_1}{2} \\
                W' &= W
            \end{align*}

            Mit Spannungsquelle werden einfach beide Kondensatoren aufgeladen, sodass:
            \begin{align*}
                W_1 &= W_1' = W_2'\\
                W' &= 2W\\
            \end{align*}

            \item 
            Die Parallel-Schaltung in (c) kann so interpretiert werden, dass
            sich ein neuer Kondensator bildet, dessen Kapazität die Summe der Kapazitäten
            der beiden einzelnen Kondensatoren ist. In diesem Sinne handelt es sich 
            bei (b) und (c) um mathematisch äquivalente Aufbauten. Für eine 
            Reihen-Schaltung würde dies nicht mehr stimmen.

        \end{enumerate}
\end{enumerate}

\end{document}