\newcommand{\ConstExercise}{3}
\newcommand{\ConstDeadline}{28.04.2023}

\documentclass[11pt,letterpaper]{article}
\textwidth 6.5in
\textheight 9.in
\oddsidemargin 0in
\headheight 0in

\usepackage[dark, exp]{custom_0.1}

\begin{document}
%%%%% Document %%%%%

\begin{enumerate}
    \item \textbf{Wärmestrahlung}\\
    Für die Emission des Kupferballs gilt:
    \begin{align*}
         \sigma A T^4 &= P = -\derivative{E}{t} = -\frac{c M\di T}{\di t}\\
        \derivative{T}{t} &= -\frac{ \sigma A}{c M}  T^4\\
        &= -\frac{ \sigma \cdot 4\pi r^2}{c\cdot \rho \cdot \frac{4}{3}\pi r^3}  T^4\\
        &= -3\frac{ \sigma}{c \rho r}  T^4
    \end{align*}
    Aufgrund des thermischen Gleichgewichts des Behältnisses gilt:
    \begin{align*}
        0 &= \dot{E}_E + \dot{E}_A\\
        &= -\varepsilon \sigma A T_B^4 + \alpha A S\\
        S &= \frac{\varepsilon}{\alpha}\sigma T_B^4\\
        &= \sigma T_B^4
    \end{align*}
    Damit gilt für die Absorption des Kupferballs:
    \begin{align*}
        \derivative{E}{t} &= AS\\
        \frac{c M\di T}{\di t} &= A \sigma T_B^4\\
        \dot{T} &= \frac{A \sigma T_B^4}{c M}\\
        &= \frac{4\pi r^2 \cdot \sigma T_B^4}{c \cdot \rho\frac{4}{3}\pi r^3}\\
        &= 3 \frac{ \sigma }{c \rho r} T_B^4
    \end{align*}
    Insgesamt:
    \begin{align*}
        \dot{T} &= \dot{T}_E + \dot{T}_A\\
        &= 3\frac{\sigma}{c \rho r} \cbrace{T_B^4
        -T^4}
        \\
        &\approx 3\frac{\csb}{383\ufrac{J}{K\,kg} \cdot  8920\, \ufrac{kg}{m^3} \cdot \frac{5}{100}\,\mathrm{m}} \cbrace{(30\ ^\circ\mathrm{C})^4
        - (5\ ^\circ\mathrm{C})^4}\\
        &\approx 8.20\cdot 10 ^{-4} \ufrac{K}{s}
    \end{align*}

    \item \textbf{Arbeit und Leistung}
        \begin{enumerate}
            \item 
            \begin{align*}
                \abs{Q _{ab}} &= 4\,\mathrm{J}\\
                \abs{Q _{zu}} &= \cbrace{1-\eta} \abs{Q_{ab}}\\
                &= \cbrace{1-\frac{17}{100}} 4\,\mathrm{J}\\
                &\approx 3.32 \,\mathrm{J}\\
                W &= \eta  \abs{Q _{ab}}\\
                &= \frac{17}{100}\cdot 4\,\mathrm{J}\\
                &\approx 0.68 \,\mathrm{J}\\
            \end{align*}

            \item 
            \begin{align*}
                P &= W \cdot f\\
                &\approx 0.68 \,\mathrm{J} \cdot 10\,\ufrac{1}{s}\\
                &\approx 6.80 \,\mathrm{W}
            \end{align*}
            
        \end{enumerate}

    \item \textbf{Erster Haubtsatz}
        \begin{enumerate}
            \item 
            \begin{align*}
                \Delta U_{ACB} &= \Delta Q_{ACB} - \Delta W_{ACB}\\
                &= 220 \,\mathrm{J} - 60 \,\mathrm{J}\\
                &= 160 \,\mathrm{J}
            \end{align*}

            \item 
            \begin{align*}
                \Delta Q_{ADB} &= \Delta U_{ADB}  + \Delta W_{ADB}\\
                &= \Delta U_{ACB}  + \Delta W_{ADB}\\
                &= 160\,\mathrm{J} + 20 \,\mathrm{J}\\
                &= 180 \,\mathrm{J}\\
            \end{align*}

            \item 
            \begin{align*}
                \Delta U_{BA} &= \Delta Q_{BA} + \Delta W_{BA}\\
                -\Delta U_{ACB} &= \Delta Q_{BA} + \Delta W_{BA}\\
                \Delta Q_{BA} &= -\Delta U_{ACB} - \Delta W_{BA}\\
                &= -160 \,\mathrm{J} - 40 \,\mathrm{J}\\
                &= -200 \,\mathrm{J}
            \end{align*}

            \item 
            \begin{align*}
                \Delta U _{AD} &= \Delta Q - \Delta W\\ 
                &= \frac{3}{2} N k_B \Delta T -p\Delta V\\
                &= \frac{3}{2} N k_B \cdot \frac{p \Delta V}{N k_B} -p\Delta V\\ 
                &= \frac{3}{2}p \Delta V -p\Delta V\\ 
                &= \Delta Q - \frac{2}{3}\Delta Q\\ 
                \Delta Q &= 3 \Delta U _{AD}\\
                &= 3\cdot 40\,\mathrm{J}\\
                &= 120 \,\mathrm{J}
            \end{align*}
            \begin{align*}
                \Delta Q _{AB} &= -\Delta Q _{BA}\\
                &= 200 \,\mathrm{J}
            \end{align*}
            
        \end{enumerate}

    \item \textbf{Ideales Gas}
        \begin{enumerate}
            \item 
            \begin{align*}
                \Delta U &= \frac{3}{2} n n_A k_B \Delta T\\
                &\approx \frac{3}{2} \frac{1}{2} \,\mathrm{mol}\cdot \ca\cdot \cb \cdot375\,\mathrm{K}\\
                &\approx 2340 \,\mathrm{J}\\
                \Delta Q &= 0 \\
                \Delta W &= \Delta U\\
            \end{align*}

            \item 
            \begin{align*}
                \Delta U &= \frac{3}{2} n n_A k_B \Delta T\\
                &\approx 2340 \,\mathrm{J}\\
                \Delta Q &= 2500\,\mathrm{J}\\
                \Delta W &= \Delta U - \Delta Q\\
                &\approx 2340 \,\mathrm{J} - 2500\,\mathrm{J}\\
                &= -160\,\mathrm{J}
            \end{align*}

            \item 
            \begin{align*}
                \Delta U &= \frac{3}{2} n n_A k_B \Delta T\\
                &\approx 2340 \,\mathrm{J}\\
                \Delta W &= -p\Delta V\\
                &= -n n_A k_B \Delta T\\
                &\approx -1560 \, \mathrm{J}\\
                \Delta Q &= \Delta U - \Delta W\\
                &= \frac{5}{2} n n_A k_B \Delta T\\
                &\approx 3900\,\mathrm{J}\\
            \end{align*}

            \item 
            \begin{align*}
                \Delta U &= \Delta Q\\
                \Delta Q &= \frac{3}{2} n  n_A k_B \Delta T\\
                &\approx 2340 \,\mathrm{J}\\
                \Delta W &= 0\\
            \end{align*}
            
        \end{enumerate}

    \item \textbf{Stirling-Prozess}
        \begin{enumerate}
            \item 
            \begin{align*}
                0 &= \Delta U_{AB} + \Delta U_{BC} + 
                \Delta U_{CD} + \Delta U_{DA}\\
                &= \frac{3}{2}N k_B \cbrace{\Delta T_{AB} + \Delta T_{BC} +
                \Delta T_{CD} + \Delta T_{DA}}\\
                &= \frac{3}{2}N k_B \cbrace{0 + \Delta T_{BC} +
                0 + \Delta T_{DA}}\\
                &= \Delta U_{BC} + \Delta U_{DA}\\
                \abs{\Delta U_{BC}} &= \abs{\Delta U_{DA}}
            \end{align*}

            \item 
            \begin{align*}
                \Delta W_1 &= -p_1 \Delta V\\
                &= -\frac{k_B N T_1}{V} \Delta V\\
                \\
                W_1 &= k_B N T_1 (\ln {V_1} - \ln{V_2}) \\
                &= k_B N T_1 \ln {\frac{V_1}{V_2}} \\
                \\
                \Delta Q _2 &= -\frac{3}{2}N k_B \Delta T \\
                \\
                W_2 &= k_B N T_2  {\frac{V_2}{V_1}} \\
                \\
                \Delta Q _4 &= - \Delta Q_2 \\
                \end{align*}
                \begin{align*}
                \eta_1 &= -\frac{W}{\Delta Q}\\
                &= \frac{W_1 + W_2}{W_1 + Q_2}\\
                &= \frac{k_B N T_1 \ln {\frac{V_1}{V_2}} - k_B N T_2  {\frac{V_1}{V_2}}}
                {k_B N T_1 \ln {\frac{V_1}{V_2}} - \frac{3}{2} N k_B (T_1 -T_2)}\\
                &= \frac{  T_1  -T_2 }
                {T_1  + \frac{3}{2} \frac{T_1 - T_2}{\ln {\frac{V_2}{V_1}}}}\\
                \\
                \eta_2 &= -\frac{W}{\Delta Q}\\
                &= \frac{W_1 + W_2}{W_1}\\
                &= \frac{k_B N T_1 \ln {\frac{V_1}{V_2}} - k_B N T_2  {\frac{V_1}{V_2}}}
                {k_B N T_1 \ln {\frac{V_1}{V_2}}}\\
                &= 1 - \frac{T_2}{T_1}\\
                \\
                \eta _1  &< \eta_2
            \end{align*}

            \item 
            \begin{align*}
                \eta_{Carnot} &=  1 - \frac{T_2}{T_1}\\
                \eta _1  &< \eta_2 = \eta_{Carnot}\\
            \end{align*}
            
        \end{enumerate}
\end{enumerate}
\end{document}