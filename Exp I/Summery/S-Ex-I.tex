\documentclass[twocolumn, bfvec]{summery_5.0}
\title{Experimentalphysik I - Zusammenfassung}
\date{WS 22/23}

\begin{document}
\maketitle
\tableofcontents

\section{Newtonsche Mechanik}
\subsection{Newtonsche Axiome}
\begin{description}
    \item[1. Axiom - Trägheitsprinzip:]\,\\
    Ein kräftefreier Körper in einem Inertialsystem, bleibt in Ruhe oder bewegt sich geradlinig mit konstanter Geschwindigkeit. 
    \item[2. Axiom - Aktionsprinzip:]\,\\
    Wirkt eine Kraft \(\v F\) auf einen Körper, so erfährt er eine Beschleunigung gemäß

    \hfill\(\v F = m \v a\)\hspace*\fill

    \item[3. Axiom - Reaktionsprinzip:]\,\\
    Eine Kraft von Körper A auf Körper B geht immer mit einer gleich großen, aber entgegen gerichteten Kraft von Körper B auf Körper A einher:

    \hfill$\v F_{A\to B} = -\v F_{B\to A}$\hspace*\fill

    \item[4. Axiom - Superpositionsprinzip:]\,\\
    Kräfte überlagern sich wie Vektoren in der Mathematik, sodass gilt:

    \hfill\(\v F = \displaystyle \sum_i \v F_i\)\hspace*\fill\,
\end{description}

\subsection{Gravitationskraft}
Das Potenzial \(\Phi\) und das Gravitationsfeld \(g\) sind mathematische Hilfsgrößen zur Bestimmung des Potenzial \(V\) und der Kraft \(F\). \(G\) ist die Gravitationskonstante. 
\begin{align*}
    \Phi(\v r) &= - G \int_{\R^3}\dV' \frac{\rho(\v r')}{\abs{\v r - \v r'}}\\
    \v g(\v r) &=  - \grad \Phi(\v r) \\
    &= G \int_{\R^3}\dV' \rho(\v r)\frac{\v r - \v r'}{\abs{\v r - \v r'}^3}\\
    \\
    V(\v r) &= m \Phi(\v r)\\
    \v F (\v r) &= m g 
\end{align*}

\subsection{Reibungskräfte}
Es wird $F_N$ verwendet als Normalkraft.
\begin{table}[H]
\begin{tabular}{@{}l c l @{}}
    \toprule
    Name & Formel & Anmerkung\\
    \midrule
    Gleitreibung & \(\v F\sub{GR}=-\mu_G \absv{F_N} \e_v\) & $\mu_G<\mu_H$\\
    Haftreibung & \(F\sub{HR} \le \mu_H \absv{F_N}\) & $\tan\alpha = \mu_H$\\
    Rollreibung & \(R\sub{RR} = - \mu_R \absv{F_N} \e_v\) & $\mu_R \propto 1/r$\\
    Stoke'sche Reibung & \(F\sub{SR} = - \kappa \v v\) & Kugel: \(\kappa = 6\pi r \eta\)\\
    Newton'sche Reibung & \(F\sub{NR} = -\xi \v v^2 \e_v\) & \(\xi =\frac12 c_w A \rho\)\\
    \bottomrule
\end{tabular}
\end{table}

\subsection{Arbeit und Energie}
Die Arbeit ist als das Wegintegral über die Kraft definiert:
\begin{align*}
    W\sub{AB} &= - \int_{K\sub{AB}} \v F(\v x) \dv x
\end{align*}
Für eine konservatives Potenzial gilt:
\begin{align*}
    W\sub{ab} &= V(\v x\sub B) - V(\v x\sub A)
\end{align*}

\subsection{Leistung}
Die Leistung $P$ ist definiert als die zeitliche Ableitung der Arbeit.
\begin{align*}
    P = \dd Wt
\end{align*}
Für die, von einem konservativen Potenzial geleisteter Arbeit, kann man auch schreiben:
\begin{align*}
    P = \dd W{\v x} \dd{\v x} t &= - \v F \cdot \v v 
\end{align*}

\section{Erhaltungssätze}
In einem abgeschlossenen System gilt die Gesamtimpulserhaltung un der Schwerpunktssatz:
\begin{align*}
    \dotv P = \sum_i F_i = 0 \tand \dotv X = \frac1M \sum_i p_i = \const 
\end{align*}
Außerdem ist die Energie erhalten. 

\section{Schwingungen}
\subsection{harmonische Schwingung}
\begin{description}
    \item[DGL:] Mit \(\gamma\) als Dämpfung und \(\omega_0\) als Kreisfrequenz des ungedämpften Oszillators.
    \begin{align*}
        \ddot x + 2\gamma \dot x + \omega_0^2 x = 0
    \end{align*}
    \item[Lösung:] Mit Fouriertransformation
    \begin{align*}
        0 &= \ddot x + 2\gamma \dot x + \omega_0^2 x\\
        &= (-ik)^2 x + 2\gamma (-ik) x + \omega_0^2 x\\
        &= k^2 + 2 i \gamma k - \omega_0^2 \\
        k_\pm &= - i \gamma \pm \underbrace{\sqrt{\omega_0^2 - \gamma^2}}_{\equiv \omega} \\
        x(t) 
        &= A e^{i k_+ t} + B e^{i k_- t}\\
        &= e^{-\gamma t}\hug{A e^{i \omega t} + B e^{-i \omega t}}
    \end{align*}
    \item[Kriechfall \(\gamma^2>\omega_0^2\):] Im Kriechfall findet keine Schwingung statt, und die Amplitude geht exponentiell gegen null.
    \item[Aperiodischer Grenzfall \(\gamma^2 = \omega_0^2\):] Die kleinste Dämpfung bei der keine Schwingung zustande kommt. Das Pendel strebt am schnellsten gegen den Nullpunkt.
    \item[Schwache Dämpfung \(\gamma^2<\omega_0^2\):] Es findet eine sinusförmige Schwingung statt, die exponentiell gedämpft ist. 
    \item[Ungedämpft \(\gamma=0\):] Eine Sinusförmige Schwingung.    
\end{description}


\section{Wellen}
Die DGL der Welle ist 
\begin{align*}
    \square \Phi &= 0 = \Delta \Phi  - \frac{1}{v^2}\p_t^2 \Phi.
\end{align*}
Die allgemeine Lösung ist eine integrale Überlagerung von Wellenfronten:
\begin{align*}
    \Phi(\v x, t) &= \int\tilde f(\v k) e^{i(\v k\cdot \v x - \omega t)} \d{^3\v k} 
\end{align*}
mit den Beziehungen
\begin{align*}
    c = \frac{\omega}{\absv k } \tand k=\frac{2\pi}\lambda
\end{align*}

\subsection{Schallwellen}
Für eine Schallwelle lässt sich die folgenden DGL herleiten:
\begin{align*}
    \Delta \Phi &= \frac{\rho_0}{p_0} \p_t^2 \Phi
\end{align*}
Wobei für Luft gilt, dass \(\sqrt{p_0/\rho_0} \approx \SI{340}{m/s}\)
und für Festkörper statt \(p_0\) das Elastizitätsmodul eingesetzt wird. 

\subsection{Überlagerung von Wellen}
Da die DGL der Wellen linear ist, gilt auch hier das Superpositionsprinzip. Beim Überlagern von Wellen entstehen viele charakteristische Effekte/Eigenschaften wie die Schwebung oder die Interferenz.

\subsection{Dopplereffekt}
Der Dopplereffekt entsteht dadurch, dass Wellen durch die Bewegung eines Senders gestaucht werden, oder ein Empfänger sich durch eine Welle bewegt. $c$ meint hier nicht die Lichtgeschwindigkeit, sondern i.A. die Ausbreitungsgeschwindigkeit der Welle. 

\begin{description}
    \item[Bewegte Quelle:] 
    \begin{align*}
        \te{Ansatz: } \lambda &= \lambda_0 - v_Q\cdot T, \ c=\const\\
        \implies  \omega &= \frac{\omega_0}{1-\frac c{v_Q}}
    \end{align*}
    \item[Bewegter Beobachter:] 
    \begin{align*}
        \te{Ansatz: } c &= c_0 + v_B, \ \lambda=\const\\
        \implies  \omega &= \omega_0 \cdot \hug{1+\frac{v_B}c}
    \end{align*}
    \item[Beide Effekte:]
    \begin{align*}
        \te{Ansatz: } f &= \frac{c}{\lambda} = \frac{c_0 + v_B}{\lambda_0 - v_Q T}\\
        \implies \omega &= \omega_0 \cdot \frac{c_0-v_B}{c+v_Q}
    \end{align*}
\end{description}

\section{Rotationsbewegungen}
\subsection{Die Kepler'schen Gesetzte}
\begin{description}
    \item[1. Kepler'sches Gesetz - Bahnkurven]
    \begin{quote}
        Die Planetenbahnen sind Ellipsen, in deren Brennpunkt die Sonne steht.
    \end{quote}
    Tatsächlich ist die Bahn nur gebunden, d.h. ein Kreis oder eine Ellipse, wenn die Gesamtenergie negativ ist:
    \begin{align*}
        E = \frac{1}{2} mv^2 - G \frac{mM}{r}<0
    \end{align*}
    Für \(E>0\) sind die Bahnkurven Parabeln und Hyperbeln.

    \item[2. Kepler'sches Gesetz - Flächensatz]
    \begin{quote}
        Der Radiusvektor \(\v r\) zwischen Sone und Planet überschreitet in gleichen Zeiten gleiche Flächen.
    \end{quote}  
    Ursache ist die Drehimpulserhaltung, denn das infinitessinale Flächenelement \(\dA\) ist gegeben durch: 
    \begin{align*}
        \dA &= \frac12 \abs{\v r \times \dv s}  = \frac12 \abs{\v r \times \v p} \frac\dt m  \\
        \dd At &= \frac{\absv L }{2m} 
    \end{align*}

    \item[3. Kepler'sches Gesetz - Perioden und Halbachsen]
    \begin{quote}
        Die Perioden \(T\) und großen Halbachsen \(a\) erfüllen \(\frac{T^2}{a^3} = \const\).
    \end{quote}
    Für eine Kreisbahn kann der Satz durch gleichsetzten von Gravitationskraft und Zentripetalkraft bewiesen werden:
    \begin{align*}
        G\frac{mM}{r^2} &= m\omega^2 r\\
        \frac{T^2}{r^3} &= \frac{4\pi^2}{GM} = \const
    \end{align*}
\end{description}

\subsection{Scheinkräfte in Beschleunigten Bezugssystemen}

Für ein beschleunigtes Bezugssystem kann man mittels einer Koordinatentransformation herleiten:
\begin{align*}
    m\v a &= \v K \underbrace{-2m\v \omega \times \dotv y}\sub{Corioliskraft} \underbrace{- m\dotv \omega \dot y}\sub{Euler-Kraft} \underbrace{-m\v \omega\times(\v \omega\times\v y)}\sub{Zentrifugalkraft} 
\end{align*}

\section{Dynamik starrer Körper}
Man unterschiedet bei Bewegungen starrer Körper zwischen Rotationen und Translationen. Eine Translation lässt sich immer mittels seines Schwerpunktes beschreiben.

\subsection{Größen der Bewegung}
In Analogie zu Größen für Translationen findet man für Rotationen:
\begin{center}
\begin{tabular}{@{}llll@{}}
    \toprule
    Größe & $\v x$ & $\v \varphi$  &Formel\\
    \midrule
    Strecke/Winkel & $\v x$ & \(\v \varphi\) & $ $ \\
    Winkel-/ Geschw. & $\v v$ & \(\v \omega\) & $\dd{\v\varphi} t = \frac{\v r \times \v v }{r^2}$ \\
    Winkel-/ Beschl. & $\v a$ & \(\dotv \omega\) & $ \dd {\v \omega}t$ \\
    Masse/Trägheitsm. & $m$ & \(I\) & $ m\v R^2 $ \\
    Trägheitstensor &  & \(\Theta\) & $ m(\v x^2\I - \v x\otimes \v x) $ \\
    Dreh-/ Impuls & $\v p$ & \(\v L\) & $ \v r \times \v p = \Theta \v \omega $ \\
    Kraft/Drehmoment & $\v F$ & \(\v M\) & $ \v r \times \v F = \Theta \dotv \omega= \dotv  L$ \\
    Kinetische Energie & $E\sub{kin}$ & \(E\sub{rot}\) & $\frac12 \v \omega \Theta \v \omega$ \\
    \bottomrule
\end{tabular}
\end{center}

\subsection{Trägheitstensor}
Der Trägheitstensor ist definiert als:
\begin{align*}
    \Theta_{ij} &= \int \rho(\v x) (\v x^2 \delta_{ij} - x_i x_j)\dV\\
    &= \int \rho(\v x) \begin{pmatrix}
        y^2+z^2 & -xy & -xz \\
        -xy & x^2 + z^2 & -yz \\
        -xz & -yz & x^2 + y^2\\
    \end{pmatrix}\dV
\end{align*}
Herleiten lässt sich dies, indem man \(\v L(\v \omega) = m \v r\times (\v \omega \times \v r)\) als Matrix \(\Omega \v \omega \) darstellt:
\begin{align*}
    \frac{\v L} m &= \v r\times \v v\\
    &= \v r\times (\v \omega \times \v r)\\
    &= \levi r^j (\varepsilon_{klm} \omega^l r^m)^k\\
    &= \levi \varepsilon_{\ lm}^k r^j \omega^l r^m\\
    &= (\delta_{il} \delta_{jm} - \delta_{im} \delta_{jl}) r^j r^m \omega^l\\
    &= (\delta_{ij} r_m r^m - r_i r_j) \omega^k\\
    \v L &= \underbrace{m (\v r^2 \I - \v r \otimes \v r) }_\Theta \cdot \v \omega
\end{align*}
Die Elemente auf der Diagonale sind die Haubtträgheitsmomente und beschreiben die Trägheit entlang der Achsen $x,y,z$. Die restlichen Elemente beschreiben die Umwuchten für diese Drehachsen.

Da der Trägheitstensor positiv definit ist, lässt er sich immer diagonalisieren.

\subsubsection{Satz von Steiner}
Man kan zeigen, dass man das Trägheitsmoment und den Trägheitstensor bei einer Verschiebung zu einer neuen Drehachse (parallel zur alten) einfach beschreiben kann durch:
\begin{align*}
    I &= I_0 + M \v a^2 \\
    \Theta_{ij} &= \Theta_{ij}^{(0)} + M(\v a^2 \delta_{ij} - a_i a_j)
\end{align*}
mit \(\v a\) als Verschiebungsvektor senkrecht zur Drehachse.

\subsubsection{Kreiseltypen}
Mit dem Vergleich der Haubtträgheitsmomente in der Diagonalform unterschiedet man zwischen drei Kreiseltypen:
\begin{enumerate}
    \item \emph{antisymmetrischer} Kreisel: \(I_a \neq I_b \neq I_c\)
    \item \emph{symmetrischer} Kreisel: \(I_a = I_b \neq I_c\)
    \item \emph{sphärischer} Kreisel: \(I_a = I_b = I_c\) 
\end{enumerate}
Die einzigen stabilen Drehungen sind solche um das kleinste oder größste Haubtträgheitsmoment.


\subsection{Komplexe Rotationsbewegungen}
Da im Allgemeinen gilt: \(\v L = \Theta \v \omega\) sind \(\v L\) und \(\v \omega\) und die Figurenachse nicht mehr zwingend parallel. Es treten verschiedene Effekte auf:

\subsubsection{Präzession}
Eine Präzession ist eine Bewegung des Drehimpulsvektors. Dieser kann (natürlich
nur wenn ein äußeres Drehmoment existiert) selbst eine Rotationsbewegung ausfüh-
ren! Wenn der Kreisel im Schwerefeld im Schwerpunkt unterstützt wird, gibt es kein Drehmoment.

Dieser Kreisel wird beginnen um die Aufhängung zu rotieren, da das Drehmoment
$M = m\v r \times \v g$ (in der Abbildung auch T ) ja in diese Richtung zeigt. man kann
sich fragen, woher kommt die Energie für diese Rotation überhaupt? Bei genauer
Betrachtung stellt man fest, dass die Konstruktion im Moment, in dem man sie
loslässt, tatsächlich etwas absackt. Dadurch ergibt sich auch eine kleine Änderung $\Delta L$ die nach unten zeigt und dann durch die Präzession kompensiert wird. Wenn man so möchte ändert sich hier die Lageenergie.


\section{Deformierbare Medien}

\section{Spezielle Relativitätstheorie}


\end{document}